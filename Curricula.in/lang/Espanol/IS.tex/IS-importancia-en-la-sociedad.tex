\section{Importancia de la carrera en la sociedad}\label{sec:cs-importancia-en-la-sociedad}

Los Sistemas de Informaci�n basados en el computador se han tornado una parte escencial de los productos, servicios, operaciones y administraci�n de las organizaciones. El uso efectivo y eficiente de las tecnolog�as de informaci�n y de comunicaci�n es un elemento importante en el logro de ventajas competitivas para las organizaciones de negocios y la excelencia en el servicio para organizaciones gubernamentales y sin fines de lucro. 

La estrategia de los sistemas de informaci�n y la tecnolog�a de informaci�n es una parte integral de la estrategia organizacional. Los sistemas de informaci�n soportan los procesos de administraci�n en todos los niveles: operacional, t�ctico y estrat�gico. As� mismo, son vitales para la identificaci�n y an�lisis de problemas as� como para la toma de decisiones. La importancia de la tecnolog�a de informaci�n y de los sistemas de informaci�n para las organizaciones y la necesidad de profesionales competentes en el campo es la base para un enlace fuerte entre los programas educacionales y la comunidad profesional de Sistemas de Informaci�n.

El camino l�gico que se espera que siga un profesional de esta �rea es que el se dedique a 
producir software, que se integre a las empresas productoras de software o al �rea de desarrollo de sistemas
dentro de cualquier organizaci�n. En Per�, la entidad que agrupa a las empresas dedicadas a la producci�n de software es la \ac{APESOFT}. Esta asociaci�n ha tomado como pol�tica principal dedicarse a la producci�n de software para exportaci�n. Siendo as�, no tendr�a sentido preparar a nuestros alumnos s�lo para el mercado local o nacional. Nuestros egresados deben estar preparados para desenvolverse en el mundo globalizado que nos ha tocado vivir.
