\begin{syllabus}

\course{MA201. C�lculo II}{Obligatorio}{MA201}

\begin{justification}
Muchas veces el comportamiento de fen�menos f�sicos, econ�micos y sociales depende de varias     variables reales continuas, el estudio de c�lculo permitir� manipular tales modelos a trav�s de las herramientas apropiadas.
\end{justification}

\begin{goals}
\item Diferenciar e integrar funciones vectoriales de variable real, entender y manejar el concepto de parametrizaci�n. Describir una curva en forma param�trica.
\item Describir, analizar, dise�ar y formular modelos continuos que dependen de m�s de una variable.
\item Establecer relaciones entre diferenciaci�n e integraci�n y aplicar el c�lculo diferencial e integral ala resoluci�n de problemas geom�tricos y de optimizaci�n.
\end{goals}

\begin{outcomes}
    \item \ShowOutcome{a}{3}
    \item \ShowOutcome{j}{3}
\end{outcomes}

\begin{competences}
    \item \ShowCompetence{C1}{a} 
    \item \ShowCompetence{C20}{j} 
    \item \ShowCompetence{C24}{j}
\end{competences}

\begin{unit}{Funci�n Real de varias Variables}{Funci�n Real de varias Variables}{espinoza2012analisis,stewart2012calculo,edward2008calculo,larson2014calculo}{18}{C1}
   \begin{topics}
      \item Funciones reales de varias variables
      \item Geometr�a de las funciones de varias variables, curvas de nivel
      \item L�mites y continuidad
      \item Derivadas parciales
      \item Diferenciabilidad, derivadas direccionales
      \item Gradiente
      \item Vectores normales y plano tangente
      \item Derivadas parciales de �rdenes superiores
      \item Derivaci�n de funciones compuestas, regla de la cadena
      \item Funciones impl�citas
   \end{topics}
   \begin{learningoutcomes}
      \item Determinar el dominio de una funci�n de varias variables[\Usage].
      \item Trazar las curvas de nivel y graficar funciones de varias variables.[\Usage].
      \item Discutir la existencia del l�mite de una funci�n en un punto[\Usage].
      \item Calcular las derivadas parciales y direccionales de funciones de varias variables.[\Usage].
      \item Determinar la ecuaci�n del plano tangente a una superficie[\Usage].
      \item Estudiar las funciones de varias variables en temas espec�ficos como son la composici�n de funciones y funciones impl�citas[\Usage].
   \end{learningoutcomes}
\end{unit}

\begin{unit}{Extremos de Funciones de varias variables}{Extremos de Funciones de varias variables}{espinoza2012analisis,stewart2012calculo,edward2008calculo,larson2014calculo,pita1995calculo}{18}{C1}
   \begin{topics}
      \item Definici�n y ejemplos preliminares
      \item Condiciones suficientes para la existencia de extremos locales
      \item Extremos condicionados. Multiplicadores de Lagrange
      \end{topics}
   \begin{learningoutcomes}
      \item Calcular los valores m�ximos y m�nimos de funciones de varias variables [\Usage].
      \item Resolver problemas de m�ximos y m�nimos [\Usage].
      \item Resolver problemas de optimizaci�n sujetos a restricciones [\Usage].
      \end{learningoutcomes}
\end{unit}

\begin{unit}{Integrales M�ltiples}{Integrales M�ltiples}{espinoza2012analisis,stewart2012calculo,edward2008calculo,larson2014calculo}{30}{C1}
   \begin{topics}
      \item Integrales dobles: funciones integrables sobre rect�ngulos
      \item Integrales dobles sobre regiones m�s generales
      \item Cambio de variables en integrales dobles
      \item Aplicaciones de las integrales dobles; vol�menes de cuerpos en el espacio, �reas de figuras planas. Masa, densidad, momentos.
      \item Integrales Triples
      \item Cambio de coordenadas en integrales triples. Aplicaciones
      \end{topics}
   \begin{learningoutcomes}
      \item Generalizar el concepto de integral definida a las funciones de varias variables [\Usage].
      \item Hallar la integral de funciones sencillas usando la definici�n de integral [\Usage].
      \item Calcular las integrales definidas en regiones m�s generales [\Usage].
      \end{learningoutcomes}
\end{unit}

\begin{unit}{Integrales de L�nea y de Superficie}{Integrales de L�nea y de Superficie}{espinoza2012analisis,stewart2012calculo,edward2008calculo,larson2014calculo}{24}{C20}
   \begin{topics}
      \item Curvas en el espacio. Parametrizaciones. Longitud de arco.
      \item Campos vectoriales.
      \item Integrales de l�nea de campos vectoriales.
      \item Integrales de l�nea de funciones reales.
      \item Integrales de superficie de funciones reales.
      \item Integrales de superficie de campos vectoriales.
      \item La rotacional de un campo vectorial. La divergencia de un campo vectorial. Laplaciano.
      \item Teoremas fundamentales: Teorema de Green y de Stokes.
      \end{topics}
   \begin{learningoutcomes}
      \item Estudiar las integrales de funciones reales y de campos vectoriales sobre curvas y superficies [\Usage].
      \item Aplicar los teoremas de Green y Stokes [\Usage].
      \end{learningoutcomes}
\end{unit}



\begin{coursebibliography}
\bibfile{BasicSciences/MA201}
\end{coursebibliography}

\end{syllabus}

%\end{document}
