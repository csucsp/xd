\begin{syllabus}

\course{FG206. Sociolog�a}{Electivos}{FG206}

\begin{justification}
La Sociolog�a permite un acercamiento vivencial a la realidad, lo que facilita su reconocimiento an�lisis y comprensi�n.\end{justification}
\begin{goals}
\item Valorar el hecho social como un factor de comprensi�n e interpretaci�n de la realidad social. [\Familiarity]
\item Interpelar la realidad desde una visi�n unitaria a partir de la perspectiva teol�gica y filos�fica. [\Familiarity]
\item Superar el planteamiento te�rico por formas de servicio desinteresado que expresen el sentido de la verdad. [\Familiarity]
\end{goals}

\begin{outcomes}
    \item \ShowOutcome{e}{1}
    \item \ShowOutcome{n}{1}
    \item \ShowOutcome{�}{2}
\end{outcomes}

\begin{competences}
    \item \ShowCompetence{C21}{e}
    \item \ShowCompetence{C22}{n} 
    \item \ShowCompetence{C24}{�}
\end{competences}

\begin{unit}{}{La Sociolog�a como Disciplina}{Macionis1999,pablo1998,Mendoza1990}{6}{C21,C22,C24}
\begin{topics}
    \item Su Naturaleza y su Objeto.
    \item Su relaci�n con las otras Ciencias.
    \item Las T�cnicas de Investigaci�n Sociol�gica.
    \item Perspectivas sociol�gicas m�s recientes.
\end{topics}
\begin{learningoutcomes}
    \item Acercarse, describir y explicar los hechos sociales. [\Familiarity]
    \item Indagar la conexi�n existente entre Sociolog�a y Sociedad. [\Familiarity]
\end{learningoutcomes}
\end{unit}

\begin{unit}{}{Cultura y Sociedad}{Pablo2001,Doig1996,Giddens2002,Mendoza1990}{9}{C21,C22,C24}
\begin{topics}
    \item Conceptos.
    \item Valores y Normas.
    \item La diversidad cultural.
    \item La socializaci�n.
    \item Los roles sociales.
    \item La identidad.
    \item Interacci�n y vida cotidiana.
    \item Tipos de sociedad.
    \item El cambio social.
    \item Grupos sociales.
    \begin{subtopics}
	    \item Primarios.
	    \item Secundarios.
    \end{subtopics}
\end{topics}

\begin{learningoutcomes}
    \item Estudiar la importancia e influencia de la cultura en la vida social humana. [\Familiarity]
    \item Analizar lo que son los grupos sociales.[\Familiarity]
    \item Entender el significado de grupo, categor�as, reuniones o aglomeraciones y organizaciones sociales. [\Familiarity]
\end{learningoutcomes}
\end{unit}

\begin{unit}{}{La familia}{Guerra2004,Benedicto2009,Morande1999}{12}{C21,C22,C24}
\begin{topics}
    \item Familia.
    \item La familia como fen�meno generalizado.
    \item Funciones insustituibles.
    \item Familia y pol�ticas p�blicas.
    \item Relaciones constitutivas.
    \item Relaciones de reciprocidad.
\end{topics}
\begin{learningoutcomes}
	\item Pensar en la realidad de la familia desde las relaciones constitutivas y de reciprocidad. Analizar su funcionalidad insustituible. [\Familiarity]
\end{learningoutcomes}
\end{unit}

\begin{unit}{}{Bien Com�n: fin y funci�n de la sociedad}{Leon2002,Pablo2010}{6}{C21,C22,C24}
\begin{topics}
    \item Los elementos que constituyen el Bien Com�n.
    \item Los ordenamientos de la Organizaci�n Social.
    \item Caracter�sticas del Bien Com�n.
    \item Principios morales del Bien Com�n.
\end{topics}
\begin{learningoutcomes}
    \item Conocer las obligaciones que tenemos con el Bien Com�n. [\Familiarity]
\end{learningoutcomes}
\end{unit}

\begin{unit}{}{Desviaci�n y control social}{Macionis1999,Giddens2002,Figari1996}{3}{C21,C22,C24}
\begin{topics}
    \item Desviaci�n, delito  y control social.
    \item Teor�as sobre el delito.
    \item El sistema de control social.
    \item El conflicto.
    \begin{subtopics}
	    \item Percepci�n.
	    \item Componentes.
    \end{subtopics}
    \item Introducci�n a la Resoluci�n del Conflicto.
\end{topics}
\begin{learningoutcomes}
    \item Analizar las normas que gu�an el rango de las actividades humanas.[\Familiarity]
\end{learningoutcomes}
\end{unit}

\begin{unit}{}{Clase, Estratificaci�n y Desigualdad}{Macionis1999,Giddens2002}{6}{C21,C22,C24}
\begin{topics}
    \item Teor�as sobre la clase y la  Estratificaci�n.
    \item La movilidad social.
\end{topics}
\begin{learningoutcomes}
	\item Distinguir la Desigualdad Social y sus principales teor�as.[\Familiarity]
\end{learningoutcomes}
\end{unit}

\begin{unit}{}{Pobreza, Bienestar y Exclusi�n Social}{Macionis1999,Giddens2002}{12}{C21,C22,C24}
\begin{topics}
    \item Pobreza.
    \begin{subtopics}
	    \item Tipos.
    \end{subtopics}
    \item Bienestar.
    \begin{subtopics}
	    \item Asistencia Social.
	    \item Estado.
    \end{subtopics}
    \item Exclusi�n Social.
    \begin{subtopics}
	    \item Formas.
    \end{subtopics}
\end{topics}
\begin{learningoutcomes}
	\item Abordar la pobreza desde dos enfoques: absoluta y relativa. [\Familiarity]
	\item Analizar la pobreza y la movilidad social. [\Familiarity]
	\item Centrar el concepto de exclusi�n social analizando sus factores condicionantes.[\Familiarity]
\end{learningoutcomes}
\end{unit}



\begin{coursebibliography}
\bibfile{GeneralEducation/FG206}
\end{coursebibliography}

\end{syllabus}
