\begin{syllabus}

\course{CS361. T�picos en Inteligencia Artificial}{Electivo}{CS361}

\begin{justification}
Provee una serie de herramientas para resolver problemas que son dif�ciles de solucionar con los m�todos algor�tmicos tradicionales. Incluyendo heur�sticas, planeamiento, formal�smos en la representaci�n del conocimiento y del razonamiento, t�cnicas de aprendizaje en m�quinas, t�cnicas aplicables a los problemas de acci�n y reacci�n: asi como el aprendizaje de lenguaje natural, visi�n artificial y rob�tica entre otros. 
\end{justification}

\begin{goals}
\item Realizar alg�n curso avanzado de Inteligencia Artificial sugerido por el curriculo de la ACM/IEEE.
\end{goals}

\begin{outcomes}
\ExpandOutcome{a}{1}
\ExpandOutcome{b}{1}
\ExpandOutcome{h}{1}
\ExpandOutcome{i}{1}
\ExpandOutcome{j}{1}
\ExpandOutcome{l}{1}
\end{outcomes}

\begin{itemize}
\item CS360. Sistemas Inteligentes
\item CS361. Razonamiento automatizado
\item CS362. Sistemas Basados en Conocimiento
\item CS363. Aprendizaje de Maquina \cite{Russell03},\cite{Haykin99}
\item CS364. Sistemas de Planeamiento
\item CS365. Procesamiento de Lenguaje Natural
\item CS366. Agentes
\item CS367. Rob�tica
\item CS368. Computaci�n Simb�lica
\item CS369. Algoritmos Gen�ticos \cite{Goldberg89}
\end{itemize}

\begin{coursebibliography}
\bibfile{Computing/CS/CS261T}

\end{coursebibliography}

\end{syllabus}
