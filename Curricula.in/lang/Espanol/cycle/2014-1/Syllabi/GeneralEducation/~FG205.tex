\begin{syllabus}

\course{HU205. Historia de la Cultura}{Obligatorio}{FG205}

\begin{justification}
Asignatura b�sica de car�cter formativo y human�stico.  Su contenido debe ser parte de la cultura fundamental de un universitario.
\end{justification}

\begin{goals}
\item \OutcomeFH
\end{goals}

\begin{outcomes}
\ExpandOutcome{FH}{2}
\ExpandOutcome{HU}{3}
\end{outcomes}

\begin{unit}{La Civilizaci�n Antigua}{KREBS,HERNADEZ,CHAPIN}{5}{2}
\begin{topics}
	\item El mundo Hel�nico / El Mundo Romano
\end{topics}
\begin{unitgoals}
	\item Conocer las bases de la cultura occidental.
\end{unitgoals}
\end{unit}

\begin{unit}{El Cristianismo}{KREBS,HERNADEZ,CHAPIN}{5}{2}
\begin{topics}
	\item La Romanidad y la Iglesia: pilares b�sicos de la civilizaci�n occidental.
\end{topics}
\begin{unitgoals}
	\item Estudiar el aporte del cristianismo a la humanidad.
\end{unitgoals}
\end{unit}

\begin{unit}{La Cultura Medieval}{KREBS,HERNADEZ,CHAPIN}{5}{2}
\begin{topics}
	\item Surgimiento y desarrollo de la edad Media
\end{topics}
\begin{unitgoals}
	\item Conocer las principales caracter�sticas de la edad Media y su papel en la formaci�n de la cultura occidental.
\end{unitgoals}
\end{unit}

\begin{unit}{El Renacimiento}{KREBS,HERNADEZ,CHAPIN}{5}{2}
\begin{topics}
	\item El renacimiento y el nacimiento de la imagen moderna del mundo.
\end{topics}
\begin{unitgoals}
	\item Conocer los hechos y las ideas que produjeron una renovaci�n del pensamiento.
\end{unitgoals}
\end{unit}

\begin{unit}{La Revuelta Protestante}{KREBS,HERNADEZ,CHAPIN}{5}{2}
\begin{topics}
	\item La �poca de las revueltas / La Reforma Cat�lica.
\end{topics}
\begin{unitgoals}
	\item Analizar la aparici�n de grupos  que quebraron la unidad cristiana y la consecuente Reforma.
\end{unitgoals}
\end{unit}

\begin{unit}{La Ilustraci�n}{KREBS,HERNADEZ,CHAPIN}{5}{2}
\begin{topics}
	\item La ilustraci�n y el endiosamiento de la raz�n.
\end{topics}
\begin{unitgoals}
	\item Valorar las nuevas ideas desarrolladas por este movimiento cultural y sus repercusiones.
\end{unitgoals}
\end{unit}

\begin{unit}{La Cuesti�n Social}{KREBS,HERNADEZ,CHAPIN}{7}{2}
\begin{topics}
	\item La revoluci�n industrial y las transformaciones pol�tico-sociales.
	\item La tecnolog�a industrial y los descubrimientos cient�ficos.
	\item Doctrina Social de la Iglesia.
\end{topics}
\begin{unitgoals}
	\item Analizar el surgimiento del problema social y la repuesta de la Iglesia.
\end{unitgoals}
\end{unit}

\begin{unit}{El Mundo Contempor�neo}{KREBS,HERNADEZ,CHAPIN}{8}{2}
\begin{topics}
	\item El Mundo Contempor�neo y el modernismo.
	\item El comienzo de la crisis del siglo XX:  guerra y revoluci�n.
	\item La infructuosa b�squeda de una nueva estabilidad: Europa entre las guerras 1919-1939.
	\item La profundidad de la crisis europea: La Segunda Guerra Mundial.
	\item La Guerra Fr�a y la nueva Europa.
	\item El mundo occidental contempor�neo a partir de 1970.
\end{topics}
\begin{unitgoals}
	\item Analizar algunos de los cambios ocurridos en nuestro tiempo.
\end{unitgoals}
\end{unit}

\begin{coursebibliography}
\bibfile{GeneralEducation/FG101}
\end{coursebibliography}

\end{syllabus}
