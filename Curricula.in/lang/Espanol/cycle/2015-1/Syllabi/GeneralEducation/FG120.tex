\begin{syllabus}

\course{FG120. Constituci�n y Realidad Nacional}{Obligatorio}{FG120}

\begin{justification}
La naturaleza de la asignatura radica en conocer los aspectos econ�micos pol�ticos y socio cultural de nuestra realidad nacional y al mismo tiempo brindar informaci�n de los acontecimientos mas resaltantes a lo largo de la historia peruana. El contenido de la asignatura se estructura de la siguiente manera: Aspectos Generales. El Estado. La Poblaci�n. El Per� y su Realidad Hist�rico-Pol�tica. La Problem�tica Social. La Problem�tica Educativa.
\end{justification}

\begin{goals}
\item Que el alumno entienda el contexto nacional sobre el cual tendra efecto su ejercicio profesional.
\item Que el alumno entienda el contexto legal existente sobre el cual ejercer� su profesi�n.
\end{goals}

\begin{outcomes}
\ExpandOutcome{HU}{2}
\end{outcomes}

\begin{unit}{Aspectos Generales}{Quijano92}{12}{2}
\begin{topics}
	\item An�lisis Coyuntural
  	\item La Realidad Social
  	\item La Realidad Econ�mica
  	\item La Realidad Pol�tica y Geogr�fica
\end{topics}

\begin{unitgoals}
      \item  Adquirir informaci�n b�sica acerca de nuestro pasado hist�rico para una reflexi�n anal�tica de nuestra realidad nacional.
   \end{unitgoals}
\end{unit}

\begin{unit}{El Estado}{Quijano92,Kapsoli93}{12}{2}
\begin{topics}
	\item El Estado
	\item Funciones del Estado Estado y Gobierno
	\item La Ciudadan�a
	\item Deberes y Derechos del ciudadano
\end{topics}

\begin{unitgoals}
      \item Describir los diferentes aspectos de la problem�tica nacional.
      \item Desarrollan una serie de actividades din�micas para una mejor comprensi�n de la realidad nacional.
   \end{unitgoals}
\end{unit}

\begin{unit}{La Poblaci�n}{Marticona93,Mariategui91}{12}{2}
\begin{topics}
	\item La poblaci�n en el Per�
	\item Distribuci�n espacial de la poblaci�n Migraciones
	\item Realidad ind�gena peruana
	\item La poblaci�n en la actividad econ�mica
\end{topics}

\begin{unitgoals}
      \item Conocer como esta ubicada la poblacion y cual es la actividad economica.
   \end{unitgoals}
\end{unit}

\begin{unit}{El Peru y su realidad historico pol�tica}{Kapsoli93}{6}{2}
\begin{topics}
	\item La Rep�blica y sus coyunturas gobiernistas.
	\item El Primer Congreso Constituyente
	\item La Reconstrucci�n Nacional
	\item El Tercer Militarismo
\end{topics}

\begin{unitgoals}
      \item Tener conocimiento del Peru y su realidad historica
\end{unitgoals}
\end{unit}

\begin{unit}{Aspestoc Sociales}{Kapsoli93}{6}{2}
\begin{topics}
	\item Identidad
	\item La Nacionalidad
	\item La pol�tica social del Per�
	\item Entre la democracia y la dictadura
	\item Gobierno Revolucionario
	\item Gobierno Democr�tico
\end{topics}

\begin{unitgoals}
      \item Consolidar los conocimientos de la politica social del Per�
\end{unitgoals}
\end{unit}



\begin{coursebibliography}
\bibfile{GeneralEducation/FG120}
\end{coursebibliography}

\end{syllabus}
