\newcommand{\DocumentVersion}{2011}
\newcommand{\fecha}{\today}
\newcommand{\city}{Santiago de Chile\xspace}
\newcommand{\country}{Chile\xspace}
\newcommand{\dictionary}{Espa�ol\xspace}
\newcommand{\GraphVersion}{2\xspace}

\newcommand{\CurriculaVersion}{2\xspace} % Malla 2006: 1
\newcommand{\YYYY}{2012\xspace}          % Plan 2006
\newcommand{\Range}{1-10}                 %Plan 2010 1-6, Plan 2006 5-10
\newcommand{\Semester}{2012-2\xspace}

% convert ./fig/UCSP.jpg ./html/img3.png
% cp ./fig/big-graph-curricula.png ./html/img18.png
% cp ./fig/small-graph-curricula.png ./html/img19.png 

\newcommand{\OutcomesList}{a,b,c,d,e,f,g,h,i,j,k,l,m,HU,FH,TASDSH}
\newcommand{\logowidth}{7cm}

\newcommand{\University}{Universidad de Chile\xspace}
\newcommand{\InstitutionURL}{http://www.uchile.cl\xspace}
\newcommand{\underlogotext}{}
\newcommand{\FacultadName}{Facultad de Ciencias F�sicas y Matem�ticas\xspace}
\newcommand{\DepartmentName}{Ciencias de la Computaci�n\xspace}
\newcommand{\SchoolFullName}{Escuela de Ingenier�a Civil en Computaci�n}
\newcommand{\SchoolFullNameBreak}{\FacultadName\\ Carrera de Ingenier�a \\Civil en Computaci�n\xspace}
\newcommand{\SchoolShortName}{Civil en Computaci�n\xspace}
\newcommand{\SchoolAcro}{ICC\xspace}
\newcommand{\SchoolURL}{http://www.dcc.uchile.cl}

\newcommand{\GradoAcademico}{Licenciado en Ciencias de la Ingenier�a (Menci�n Computaci�n)\xspace}
\newcommand{\TituloProfesional}{Ingeniero Civil en Computaci�n\xspace}
\newcommand{\GradosyTitulos}%
{\begin{description}%
\item [Grado Acad�mico: ] \GradoAcademico\xspace y%
\item [Titulo Profesional: ] \TituloProfesional%
\end{description}%
}

\newcommand{\doctitle}{Plan de Estudios \YYYY\xspace del \SchoolFullName\\ \SchoolURL}

\newcommand{\AbstractIntro}{Este documento presenta el informe preliminar de la nueva malla 
curricular \YYYY del \SchoolFullName de la \University (\textit{\InstitutionURL}) en la ciudad 
de \city-\country.}

\newcommand{\OtherKeyStones}{
Un pilar que merece especial consideraci�n en el caso de la \University es el aspecto de formaci�n en desarrollo 
del pensamiento, comunicaci�n efectiva, desarrollo personal y formaci�n ciudadana.\xspace}

\newcommand{\profile}{
El perfil de los egresados de la carrera de Ingenier�a Civil en Computaci�n (ICC) sigue los est�ndares de la iniciativa CDIO, 
que define un marco acerca de las habilidades fundamentales para los ingenieros de la pr�xima generaci�n. 
Est� orientado a un fuerte dominio de las matem�ticas, ciencias b�sicas, ciencias de la ingenier�a y comprensi�n global 
de los fundamentos de la Ciencia de la Computaci�n, incluyendo conocimiento avanzado en una o m�s �reas, y capacidad 
para aplicar estos conocimientos de una manera rigurosa e integrada. 

Los egresados de esta carrera poseen un dominio de t�cnicas y herramientas modernas necesarias para el ejercicio 
de su profesi�n, y la habilidad para concebir, dise�ar, implementar, operar, evaluar y controlar software, sistemas, 
componentes o procesos, en forma eficiente y creativa, que cumplan con las especificaciones demandadas por el contexto; 
considerando las restricciones econ�micas, ambientales, sociales, pol�ticas, �ticas, de salud y seguridad, 
de manufacturaci�n y sustentabilidad.
}

\newcommand{\HTMLFootnote}{{Generado por <A HREF='http://socios.spc.org.pe/ecuadros/'>Ernesto Cuadros-Vargas</A> <ecuadros AT spc.org.pe>, <A HREF='http://www.spc.org.pe/'>Sociedad Peruana de Computaci�n-Per�</A>, <A HREF='http://www.ucsp.edu.pe/'>Universidad Cat�lica San Pablo, Arequipa-Per�</A><BR>basado en el modelo de la {\it Computing Curricula} de <A HREF='http://www.computer.org/'>IEEE-CS</A>/<A HREF='http://www.acm.org/'>ACM</A>}}
\newcommand{\Copyrights}{Generado por Ernesto Cuadros-Vargas (ecuadros AT spc.org.pe), Sociedad Peruana de Computaci�n (http://www.spc.org.pe/), Universidad Cat�lica San Pablo (http://www.ucsp.edu.pe) Per� basado en la {\it Computing Curricula} de IEEE-CS (http://www.computer.org) y ACM (http://www.acm.org/)}
