\begin{syllabus}

\course{CS1D1. Estructuras Discretas I}{Obligatorio}{CS1D1}

\begin{justification}
Las estructuras discretas proporcionan los fundamentos te�ricos necesarios para la computaci�n.
Dichos fundamentos no son s�lo �tiles para desarrollar la computaci�n desde un punto de vista te�rico
como sucede en el curso de teor�a de la computaci�n,  sino que tambi�n son �tiles para la pr�ctica de la
computaci�n; en particular se aplica en �reas como verificaci�n, criptograf�a, m�todos formales, etc.
\end{justification}

\begin{goals}
\item Aplicar adecuadamente conceptos de la matem�tica finita (conjuntos, relaciones, funciones) para representar datos de problemas reales.
\item Modelar situaciones reales descritas en lenguaje natural, usando l�gica proposicional y l�gica de predicados.
\item Aplicar el m�todo de demostraci�n m�s adecuado para determinar la veracidad de un enunciado.
\item Construir argumentos matem�ticos correctos.
\item Interpretar las soluciones matem�ticas para un problema y determinar su fiabilidad, ventajas y desventajas.
\item Expresar el funcionamiento de un circuito electr�nico simple usando conceptos del �lgebra de Boole.
\end{goals}

\begin{outcomes}
    \item \ShowOutcome{a}{2}
    \item \ShowOutcome{i}{3}
    \item \ShowOutcome{j}{2}
\end{outcomes}

\begin{competences}
    \item \ShowCompetence{C1}{a}
    \item \ShowCompetence{C20}{i,j}
\end{competences}

 
 \begin{unit}{\DSBasicLogic}{}{Rosen2012,Grimaldi03}{14}{C1,C20}
   \begin{topics}
    % KU: L?gica b?sica
        \item \DSBasicLogicTopicPropositional%
        \item \DSBasicLogicTopicLogical%
        \item \DSBasicLogicTopicTruth%
        \item \DSBasicLogicTopicNormal%
        \item \DSBasicLogicTopicValidity%
        \item \DSBasicLogicTopicPropositionalInference%
        \item \DSBasicLogicTopicPredicate%
        \item \DSBasicLogicTopicLimitations%
   \end{topics}
   \begin{learningoutcomes}
	\item \DSBasicLogicLOConvertLogical [\Usage ]
	\item \DSBasicLogicLOApplyFormal [\Usage ]
	\item \DSBasicLogicLOUseThe [\Usage]
	\item \DSBasicLogicLODescribeHowCan [\Familiarity]
	\item \DSBasicLogicLOApplyFormalAnd [\Usage ]
	\item \DSBasicLogicLODescribeTheLimitationsAnd [\Usage]
   \end{learningoutcomes}
 \end{unit}

\begin{unit}{\DSProofTechniques}{}{Rosen2012, Epp10, Scheinerman12}{14}{C1,C20}
\begin{topics}%
	% KU: T?cnicas de demostraci?n
        \item \DSProofTechniquesTopicNotions%
        \item \DSProofTechniquesTopicThe%
        \item \DSProofTechniquesTopicDirect%
        \item \DSProofTechniquesTopicDisproving%
        \item \DSProofTechniquesTopicProof%s
        \item \DSProofTechniquesTopicInduction%
        \item \DSProofTechniquesTopicStructural%
        \item \DSProofTechniquesTopicWeak%
        \item \DSProofTechniquesTopicRecursive%
        \item \DSProofTechniquesTopicWell%
\end{topics}
\begin{learningoutcomes}
    %% itemizar cada learning outcomes [nivel segun el curso]
  \item \DSProofTechniquesLOIdentifyTheUsed [\Assessment]
  \item \DSProofTechniquesLOOutline [\Usage ]
  \item \DSProofTechniquesLOApplyEach [\Usage ]
  \item \DSProofTechniquesLODetermineWhich [\Assessment]
  \item \DSProofTechniquesLOExplainTheIdeas [\Familiarity ]
  \item \DSProofTechniquesLOExplainTheWeak [\Assessment]
  \item \DSProofTechniquesLOStateThe [\Familiarity]
\end{learningoutcomes}
\end{unit}

\begin{unit}{\DSSetsRelationsandFunctions}{}{Grimaldi03,Rosen2012}{13}{C1,C20}
   \begin{topics}
        \item \DSSetsRelationsandFunctionsTopicSets
        \item \DSSetsRelationsandFunctionsTopicRelations
        \item \DSSetsRelationsandFunctionsTopicFunctions
   \end{topics}
   \begin{learningoutcomes}
	\item \DSSetsRelationsandFunctionsLOExplainWith [\Assessment]
	\item \DSSetsRelationsandFunctionsLOPerformThe [\Assessment]
	\item \DSSetsRelationsandFunctionsLORelate [\Assessment]
   \end{learningoutcomes}
\end{unit}

\begin{unit}{Fundamentos de L�gica Digital}{}{Rosen2012,Grimaldi03}{19}{C1,C20}
   \begin{topics}
	\item �rdenes Parciales y Conjuntos Parcialmente Ordenados.
 	\item Elementos extremos de un Conjunto Parcialmente Ordenado.
	\item Ret�culas: Tipos y propiedades.
	\item �lgebras Booleanas
	\item Funciones y expresiones Boolenas
	\item Representaci�n de Funciones Booleanas: Forma Normal Disyuntiva y Conjuntiva
	\item Puertas L�gicas
	\item Minimizaci�n de funciones booleanas.
   \end{topics}
   \begin{learningoutcomes}
	\item Explicar la importancia del �lgebra de Boole como unificaci�n de la Teor�a de Conjuntos y la L�gica Proposicional [\Familiarity].
	\item Demostrar enunciados usando el concepto de ret�cula y sus propiedades[\Assessment].
	\item Explicar la relaci�n entre ret�cula y conjunto parcialmente ordenado [\Familiarity].
	\item Demostrar para una terna formada por un conjunto y dos operaciones internas, si cumple las propiedades de una �lgebra de Boole [\Assessment].
	\item Representar una funci�n booleana en sus formas can�nicas[\Usage].
	\item Representar una funci�n booleana como un circuito booleano usando puertas l�gicas  [\Usage].
	\item Minimizar una funci�n booleana [\Usage].
    \end{learningoutcomes}
 \end{unit}



\begin{coursebibliography}
\bibfile{Computing/CS/CS1D1}
\end{coursebibliography}

\end{syllabus}

%\end{document}
