\begin{syllabus}

\course{CS315. Estructuras de Datos Avanzadas}{Obligatorio}{CS315}

\begin{justification}
Los algoritmos y estructuras de datos son una parte fundamental de la ciencia de la computaci�n que nos permiten organizar la informaci�n de una manera m�s eficiente, por lo que es importante para todo profesional del �rea tener una s�lida formaci�n en este aspecto.

En el curso de estructuras de datos avanzadas nuestro objetivo es que el alumno conozca y analize estructuras complejas, como los M�todos de Acceso Multidimensional, M�todos de Acceso Espacio-Temporal y M�todos de Acceso M�trico, etc.
\end{justification}

\begin{goals}
\item Que el alumno entienda, dise�e, implemente, aplique y
proponga estructuras de datos innovadoras para solucionar
problemas relacionados al tratamiento de datos multidimensionales,
recuperaci�n de informaci�n por similitud, motores de b�squeda y
otros problemas computacionales.
\end{goals}

\begin{outcomes}
\ExpandOutcome{a}{3}
\ExpandOutcome{b}{4}
\ExpandOutcome{c}{3}
\ExpandOutcome{g}{4}
\ExpandOutcome{j}{3}
\ExpandOutcome{k}{3}
\end{outcomes}

\begin{unit}{T�cnicas B�sicas de Implementaci�n de Estructuras de Datos}{Cuadros2004Implementing,Knuth2007TAOCP-V-I,Knuth2007TAOCP-V-II,Gamma94}{16}{5}
   \begin{topics}
         \item Programaci�n estructurada
         \item Programaci�n Orientada a Objetos
         \item Tipos Abstractos de Datos
         \item Independencia del lenguaje de programaci�n del usuario de la estructura
         \item Independencia de Plataforma
         \item Control de concurrencia
         \item Protecci�n de Datos
         \item Niveles de encapsulamiento (struct, class, namespace, etc)
   \end{topics}

   \begin{unitgoals}
         \item Que el alumno entienda las diferencias b�sicas que involucran las distintas t�cnicas de implementaci�n de estructuras de datos
         \item Que el alumno analice las ventajas y desventajas de cada una de las t�cnicas existentes
   \end{unitgoals}
\end{unit}

\begin{unit}{M�todos de Acceso Multidimensionales}{Gaede98multidimensional}{16}{4}
   \begin{topics}
         \item M�todos de Acceso para datos puntuales
         \item M�todos de Acceso para datos no puntuales
         \item Problemas relacionados con el aumento de dimensi�n
         \item M�todos de Acceso Espacio-Temporales
   \end{topics}

   \begin{unitgoals}
         \item Que el alumno entienda conozca e implemente algunos M�todos de Acceso para datos multidimensionales y espacio temporales
         \item Que el alumno entienda el potencial de estos M�todos de Acceso en el futuro de las bases de datos comerciales
   \end{unitgoals}
\end{unit}

\begin{unit}{M�todos de Acceso M�trico}{Chavez:01,Traina00SlimTree,Zezula07}{20}{1}
   \begin{topics}
         \item M�todos de Acceso M�trico para distancias discretas
         \item M�todos de Acceso M�trico para distancias continuas
   \end{topics}

   \begin{unitgoals}
         \item Que el alumno entienda conozca e implemente algunos m�todos de acceso m�trico
         \item Que el alumno entienda la imnportancia de estos M�todos de Acceso para la Recuperaci�n de Informaci�n por Similitud
   \end{unitgoals}
\end{unit}

\begin{unit}{Seminarios}{Chavez:01}{8}{1}
	\begin{topics}
         \item M�todos de Acceso Espacio Temporal
         \item Estructuras de Datos con programaci�n gen�rica
   \end{topics}

   \begin{unitgoals}
         \item Que el alumno pueda discutir sobre los �ltimos avances en m�todos de acceso para distintos dominios de conocimiento
   \end{unitgoals}
\end{unit}

\begin{coursebibliography}
\bibfile{Computing/CS/CS315}
\end{coursebibliography}

\end{syllabus}
