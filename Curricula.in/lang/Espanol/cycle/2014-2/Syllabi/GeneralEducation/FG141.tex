\begin{syllabus}

\course{FG141. Historia de las Ideas}{Obligatorio}{FG141}

\begin{justification}
El aprender a aprender en la legislaci�n universitaria, es una oportunidad 
abierta a todas las corrientes y formas del pensamiento expuestas de manera 
cient�fica. La presente facilitaci�n dirige su actividad a la formaci�n integral 
del estudiante universitario para forjar en �l principio de justicia social, 
identidad institucional, pluralismo, paz, afirmar la democracia, los 
derechos humanos y la solidaridad entre todos los estamentos del convivir 
gregoriano. Se trabajar� planteamientos, t�cnicas y estrategias, que 
propicien el di�logo que difundan y fortalezcan los valores en la 
sociedad estudiantil, la formaci�n integral y el desarrollo humano, 
paralelo a la formaci�n profesional, t�cnica y cient�fica, puesto 
que resulta incompatible con los principios se�alados, la violencia, 
intolerancia y discriminaci�n, por lo que la normatividad establecida 
en la vida jur�dica de la instituci�n adopta la aplicaci�n de pol�ticas 
y mecanismos espec�ficos que promueven y garantizan el pensamiento cr�tico 
y la conciencia social en los miembros de la comunidad universitaria; 
generando una respetuosa relaci�n humana y una plena realizaci�n 
profesional y personal.
\end{justification}

\begin{goals}
\item Lograr que los estudiantes interioricen los procesos de la convivencia universitaria
\item Apropiar de estabilidad y conciencia humana a trav�s de principos y valores que requiere el estudiante
\item Lograr una concienciaci�n de los procesos de evaluaci�n que asumen docentes, estudiantes e instituci�n universitaria
\end{goals}

\begin{outcomes}
\ExpandOutcome{e}{2}
\ExpandOutcome{f}{2}
\ExpandOutcome{HU}{3}
\end{outcomes}

\begin{unit}{Introducci�n a la vida universitaria}{Asamblea08}{10}{2}
   \begin{topics}
	\item Naturaleza de la educaci�n superior
	\item Visi�n, Misi�n
	\item Fines, Objetivos
   \end{topics}

   \begin{unitgoals}
      \item Dar a conocer las generalidades de la educaci�n superior
   \end{unitgoals}
\end{unit}

\begin{unit}{Estructura org�nica funcional de la Universidad}{Gregorio08}{10}{2}
   \begin{topics}
        \item Nivel Directivo
	\item Nivel Consultivo
	\item Nivel Asesor
	\item Nivel de Apoyo
	\item Nivel Operativo
  \end{topics}

   \begin{unitgoals}
      \item Posesionar a los estudiantes del orden estructural y funcional que posee la universidad
   \end{unitgoals}
\end{unit}

\begin{unit}{Hermene�tica jur�dica universitaria}{Congreso00}{10}{2}
   \begin{topics}
        \item Ley de Educaci�n Superior 
	\item Reglamento a la Ley
	\item Estatuto
	\item Reglamento de R�gimen Acad�mico
   \end{topics}

   \begin{unitgoals}
      \item Facilitar conocimientos de organizaci�n, funcionalidad y responsabilidad que asumen las universidades a trav�s de la Ley de Educaci�n Superior
      \item Reconocer las funciones y derechos que le asisten a universidades, docentes y estudiantes
   \end{unitgoals}
\end{unit}

\begin{learning-strategies}
\FGLearningStrategies
\end{learning-strategies}

\begin{evaluation}
\FGEvaluation
\end{evaluation}

\begin{didactical-resources}
Aulas totalmente equipadas con la tecnolog�a suficiente:

\begin{inparaenum}[ \bf I:]
\item Pizarra.
\item Borrador.
\item Pilots.
\item Proyector de v�deo (v�deo-beam).
\item Computadora.
\item Laboratorio de c�mputo
\item Acceso a internet por cable modem.
\end{inparaenum}

\end{didactical-resources}

% \begin{schedule}
% \end{schedule}

\begin{coursebibliography}
\bibfile{GeneralEducation/FG122}
\end{coursebibliography}

\end{syllabus}
