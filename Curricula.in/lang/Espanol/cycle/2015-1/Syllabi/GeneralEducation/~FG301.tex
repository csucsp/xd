\begin{syllabus}

\course{FG301. Ense�anza Social de la Iglesia}{Obligatorio}{FG301}

\begin{justification}
El conocimiento y realizaci�n del Pensamiento Social de la Iglesia, es clave en el desarrollo personal 
y en la respuesta a la realidad peruana actual, buscando llegar a la construcci�n de una 
sociedad justa y reconciliada.
\end{justification}

\begin{goals}
\item \OutcomeFH
\end{goals}

\begin{outcomes}
\ExpandOutcome{FH}{2}
\end{outcomes}

\begin{unit}{Centralidad de la Persona Humana en la Cultura}{TOSO,PALUMBO}{8}{2}
\begin{topics}
	\item Naturaleza de la Doctrina Social de la Iglesia.  El Pensamiento Social de la Iglesia.
	\item La Iglesia y la sociedad.
	\item Presupuestos antropol�gicos y eclesiales
	\item Cultura, Centralidad de la persona humana en al cultura.
	\item Los ordenes social, econ�mico y pol�tico, expresi�n de la cultura en funci�n a la persona humana.  Instancias de pertenencia
\end{topics}
\begin{unitgoals}
	\item Comprender la naturaleza de la acci�n de la Iglesia en el mundo.
	\item Comprender la naturaleza del t�rmino cultura para la Iglesia.
	\item Comprender los �rdenes social, econ�mico y pol�tico insertos en al cultura.
\end{unitgoals}
\end{unit}

\begin{unit}{Perspectiva Hist�rica de la Doctrina Social de la Iglesia}{TOSO,PALUMBO}{7}{2}
\begin{topics}
	\item Fundamentaci�n b�blica, Antiguo y Nuevo Testamento, la misi�n de Jes�s, la misi�n de la Iglesia.
	\item Antropolog�a y derechos.
\end{topics}
\begin{unitgoals}
	\item Comprender que los fundamentos de al Doctrina Social de la Iglesia se inspiran en la revelaci�n.
	\item Descubrir en al historia el desarrollo de distintas acciones e instituciones como practica social de la iglesia.
\end{unitgoals}
\end{unit}

\begin{unit}{Principios y Valores de la Doctrina Social de  la Iglesia}{TOSO,PALUMBO}{7}{2}
\begin{topics}
	\item Dignidad humana.
	\item Destino universal de los bienes.
	\item Solidaridad.
	\item Subsidiaridad.
	\item Bien com�n.
	\item Pluralismo social. 
\end{topics}
\begin{unitgoals}
	\item Conocer y comprender los principios permanentes y valores fundamentales que est�n presentes en la Ense�anza Magisterial, los cuales deben ser la base para la formaci�n de las diversas instancias sociales.
\end{unitgoals}
\end{unit}

\begin{unit}{Instancias de Pertenencia: La Familia}{TOSO,PALUMBO}{8}{2}
\begin{topics}
	\item Familia, comuni�n y comunidad de vida.
	\item Comuni�n conyugal y matrimonio. 
	\item Matrimonio e indisolubilidad.
	\item Familia y sociedad.
	\item Familia, dentro de la civilizaron del amo.
	\item Familia es sociedad natural
	\item Familia necesaria para la vida social.
	\item Caracter�sticas de la familia.
\end{topics}
\begin{unitgoals}
	\item Comprender que de la naturaleza social del hombre deriva, algunos �rdenes sociales necesarios, como la familia.
	\item Conocer, comprender y valorar la naturaleza de la familia y el matrimonio y su rol en al sociedad.
\end{unitgoals}
\end{unit}

\begin{unit}{Comunidad Pol�tica}{TOSO,PALUMBO}{7}{2}
\begin{topics}
	\item Naci�n, Patria y Estado.
	\item Origen, valor, relaci�n con a sociedad civil.
	\item Elementos constitutivos del ser de la comunidad pol�tica.
	\item Autoridad
	\item Bien com�n y derechos humanos.
	\item Democracia.
	\item Iglesia y estado
\end{topics}
\begin{unitgoals}
	\item Comprender que de la naturaleza social del hombre derivan, la naci�n y el Estado como �rdenes sociales necesarios.
\end{unitgoals}
\end{unit}

\begin{unit}{Orden Econ�mico y Trabajo}{TOSO,PALUMBO}{8}{2}
\begin{topics}
	\item Aspectos b�blicos sobre los bienes, la riqueza y la actividad econ�mica.
	\item La globalizaci�n de la econom�a.
	\item Vida econ�mica
	\item Mundializaci�n de la econom�a.
	\item El padre trabaja siempre.
	\item Iglesia y nuevas caracter�sticas en el mundo del trabajo.
	\item Prioridad del trabajo sobre el capital.
	\item Deber y derecho del trabajo.
	\item La desocupaci�n
	\item Derechos de los trabajadores.
	\item La huelga.
\end{topics}
\begin{unitgoals}
	\item Conocer y comprender los principios de la Doctrina Social de la Iglesia en el campo de la actividad econ�mica.
	\item Formaci�n de la conciencia cristiana para el posterior desenvolvimiento profesional.
	\item Comprender que los principios del Evangelio y de la �tica natural pueden ser aplicados a las concreciones del orden econ�mico de la actividad humana.
\end{unitgoals}
\end{unit}

\begin{coursebibliography}
\bibfile{GeneralEducation/FG101}
\end{coursebibliography}
\end{syllabus}
