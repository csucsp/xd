\begin{syllabus}

\course{FG109. Realidad Nacional}{Obligatorio}{FG109}

\begin{justification}
Est� encaminada a conocer profundamente los elementos que constituyen la realidad de 
nuestro entorno para concientizar a los estudiantes de la situaci�n nacional, 
para dar el mayor n�mero de elementos de juicio para que el estudiante aprenda a tomar 
posiciones en las mejores condiciones de racionalidad, libertad y sociabilidad.
\end{justification}

\begin{goals}
\item Aplicar t�cnicas din�micas grupales que propician la integraci�n de los participantes, rompiendo estructuras tradicionales en el aprendizaje.
\item Desarrollar en el estudiante la capacidad anal�tica y cr�tica sobre la realidad nacional.
\item Incentivar al estudiante sobre la importancia de conocer nuestra realidad.
\end{goals}

\begin{outcomes}
\ExpandOutcome{FH}{1}
\end{outcomes}

\begin{unit}{Un pa�s pluri�tnico y pluricultural}{Almeida94}{10}{1}
   \begin{topics}
	\item Los pueblos ind�genas
	\item Poblaciones y poblamientos
	\item Los espacios etnoculturales
   \end{topics}

   \begin{unitgoals}
      \item Conocer la diversidad �tnica y cultural de nuestro pa�s
   \end{unitgoals}
\end{unit}

\begin{unit}{Los organismos del Estado}{Huerta99}{15}{1}
   \begin{topics}
      \item Funciones
	\item Atribuciones del presidente de la rep�blica
	\item �rganos de la funci�n judicial
	\item Los organismos de control
  \end{topics}

   \begin{unitgoals}
      \item Analizar las funciones de cada una de las instancias gubernmentales
   \end{unitgoals}
\end{unit}

\begin{unit}{La realidad socioecon�mica del Ecuador}{Acosta95}{10}{1}
   \begin{topics}
      \item Los espacios de bienestar y pobreza
	\item Desarrollo y subdesarrollo
	\item Las diferentes interpretaciones y propuestas de soluci�n a los problemas socioecon�micos
	\item El contexto mundial de la econom�a
   \end{topics}

   \begin{unitgoals}
      \item Enfrentar los problemas socioecon�micos derivados de nuestro entorno con la situaci�n mundial
   \end{unitgoals}
\end{unit}

\begin{unit}{Las diferentes perspectivas}{Vicu�a07}{10}{1}
   \begin{topics}
      \item Los diferentes tipos de actores econ�micos
	\item Los sectores econ�micos
	\item La producci�n para la exportaci�n y para el consumo interno
	\item La econom�a del petr�leo
	\item Las econom�as modernas y tradicionales
	\item Hacia un proyecto nacional
	\item Regiones, autonom�a, centralismo, bi-descentralizaci�n
	\item Cifras del Ecuador
	\item Emigraciones/Inmigraciones
	\item Objetivos del Milenio: inclusi�n social 
   \end{topics}

   \begin{unitgoals}
      \item Dise�ar un plan de perspectivas encaminadas a conocer posibles soluciones a los problemas de nuestro pa�s
   \end{unitgoals}
\end{unit}

\begin{coursebibliography}
\bibfile{GeneralEducation/FG109}
\end{coursebibliography}
\end{syllabus}
