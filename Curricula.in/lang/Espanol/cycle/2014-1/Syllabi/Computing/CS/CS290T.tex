\begin{syllabus}

\course{CS290T. Ingenier�a de Software I}{Obligatorio}{CS290T}

\begin{justification}
La tar�a de desarrollar software, excepto para aplicaciones sumamente simples, exige la ejecuci�n de un proceso de desarrollo bien definido. 
Los profesionales de esta �rea requieren un alto grado de conocimiento de los diferentes modelos e proceso de desarrollo, 
para que sean capaces de elegir el m�s id�neo para cada proyecto de desarrollo. Por otro lado, el desarrollo de sistemas 
de mediana y gran escala requiere del uso de bibliotecas de patrones y componentes y del dominio de t�cnicas relacionadas al 
dise�o basado en componentes.
%Proporciona una introducci�n intensiva, y orientada a la pr�ctica de las t�cnicas de desarrollo de software usadas para
%crear aplicaciones interactivas de mediana escala, centr�ndose en el uso de bibliotecas de orientaci�n a objetos
%grandes para crear interfaces de usuario bien dise�adas.
\end{justification}

\begin{goals}
\item Brindar al alumno un marco te�rico y pr�ctico para el desarrollo de software bajo est�ndares de calidad.
\item Familiarizar al alumno con los procesos de modelamiento y construcci�n de software a trav�s del uso de herramientas CASE.
\item Los alumnos debe ser capaces de seleccionar Arquitecturas y Plataformas tecnol�gicas ad-hoc a los escenarios de implementaci�n.
\item Aplicar el modelamiento basado en componentes y fin de asegurar variables como calidad, costo  y {\it time-to-market} en los procesos de desarrollo.
\item Brindar a los alumnos mejores pr�cticas para la verificaci�n y validaci�n del software.
\end{goals}

\begin{outcomes}
\ExpandOutcome{b}{4}
\ExpandOutcome{c}{4}
\ExpandOutcome{d}{3}
\ExpandOutcome{f}{3}
\ExpandOutcome{i}{3}
\ExpandOutcome{j}{3}
\ExpandOutcome{k}{3}
\end{outcomes}

\begin{unit}{\SESoftwareDesignDef}{Pressman2005,Sommerville2008,Larman2008}{12}{4}
    \SESoftwareDesignAllTopics
    \SESoftwareDesignAllObjectives
\end{unit}

\begin{unit}{\SEUsingAPIsDef}{Pressman2005,Sommerville2008}{6}{3}
   \SEUsingAPIsAllTopics
   \SEUsingAPIsAllObjectives
\end{unit}

\begin{unit}{\SEToolsAndEnvironmentsDef}{Pressman2005,Sommerville2008,Long91}{8}{3}
    \SEToolsAndEnvironmentsAllTopics
    \SEToolsAndEnvironmentsAllObjectives
\end{unit}

\begin{unit}{\SESoftwareValidationDef}{Pressman2005,Sommerville2008,Larman2008}{8}{3}
    \SESoftwareValidationAllTopics
    \SESoftwareValidationAllObjectives
\end{unit}

\begin{unit}{\SEComponentBasedComputingDef}{Pressman2005,Sommerville2008,Larman2008}{14}{3}
    \SEComponentBasedComputingAllTopics
    \SEComponentBasedComputingAllObjectives
\end{unit}

\begin{unit}{\SESpecializedSystemsDef}{Pressman2005,Sommerville2008,Larman2008}{4}{3}
    \SESpecializedSystemsAllTopics
    \SESpecializedSystemsAllObjectives
\end{unit}

\begin{unit}{\SERobustAndSecurityDef}{Pressman2005,Sommerville2008,Larman2008}{8}{3}
    \SERobustAndSecurityAllTopics
    \SERobustAndSecurityAllObjectives
\end{unit}



\begin{coursebibliography}
\bibfile{Computing/CS/CS290T}
\end{coursebibliography}

\end{syllabus}
