\begin{syllabus}

\course{CS410. Pasant�a I}{Obligatorio}{CS410}

\begin{justification}
El nivel o evento de pasantia es un espacio de formaci�n pr�ctico-te�rico-investigativo, que permitir� al pasante fortalecer y profundizar los conocimientos y operaciones que se desarrollan y se proyectan en la empresa e instituci�n en el campo de la Ciencias de la Computaci�n. El evento se constituye en un aporte cognitivo y pr�ctico de formaci�n profesional. El escenario de actuaci�n del pasante, estar� dentro de un marco de responsabilidad acad�mica, coordinado por la carrera profesional y la empresa de convenio.
\end{justification}

\begin{goals}
\item Lograr que los estudiantes fortalezcan sus conocimientos te�ricos-pr�cticos, con la verificaci�n de actividades y acciones de su pr�ctica profesional, en un espacio de observaci�n y apoyo t�cnico en el desarrollo empresarial e institucional  
\end{goals}

\begin{outcomes}
\ExpandOutcome{f}{3}
\ExpandOutcome{h}{5}
\ExpandOutcome{n}{3}
\end{outcomes}

\begin{unit}{Pasantia I}{Pasantia}{75}{4}
   \begin{topics}
      \item Observaci�n, apoyo y verificaci�n de los procesos pr�cticos-t�cnicos que desarrolla la empresa en computaci�n.
   \end{topics}
   
   \begin{unitgoals}
      \item Que el alumno tenga una experiencia en el campo laboral que le permita consolidar los conocimientos adquiridos en su carrera.
   \end{unitgoals}
\end{unit}

\begin{coursebibliography}
\bibfile{Computing/CS/CS410}
\end{coursebibliography}

\end{syllabus}
