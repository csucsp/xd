\begin{syllabus}

\course{FG102. Metodolog�a del Estudio}{Obligatorio}{FG102}

\begin{justification}
Los alumnos en formaci�n profesional necesitan comprender la exigencia del trabajo universitario y dominar las variadas formas de estudiar, para que puedan seleccionar los m�todos mas convenientes a su personal estilo de aprender y a la naturaleza de cada asignatura. De ese modo podr� aplicarlos a su trabajo universitario, haciendo exitoso se esfuerzo.
\end{justification}

\begin{goals}
\item Identificar t�cnicas de estudio adecuadas.
\item Reconocer m�todos de investigaci�n cient�fica.
\end{goals}

\begin{outcomes}
\ExpandOutcome{f}{1}
\ExpandOutcome{h}{1}
\ExpandOutcome{l}{1}
\end{outcomes}

\begin{unit}{Primera Unidad}{Bernedo}{15}{1}
\begin{topics}
	\item La formaci�n profesional y el profesional.
	\item La postura ante el reto del trabajo universitario.
	\item Organizaci�n personal y de los recursos par el estudio exitoso.
	\item La dosificaci�n del tiempo y del Esfuerzo.
	\item Las caracter�sticas del trabajo e la Universidad y las exigencias del mismo.
	\item Planes personales del mismo.
\end{topics}
\begin{unitgoals}
	\item Comprender y aceptar la formaci�n profesional como un proceso que exige trabajo intelectual consciente y organizado con enfoque el ejercicio profesional responsable.
	\item Explicar las caracter�sticas del trabajo universitario y los niveles de exigencia que esta circunstancia reclama.
\end{unitgoals}
\end{unit}

\begin{unit}{Segunda Unidad}{Flores}{15}{1}
\begin{topics}
	\item La ciencia.
	\item El m�todo en general.
	\item Los m�todos l�icos.
	\item Conocimiento y aprendizaje
	\item Expresi�n y transmisi�n del conocimiento. Elementos b�sicos de la transmisi�n oral del conocimiento.
	\item Los recursos para el aprendizaje.
\end{topics}
\begin{unitgoals}
	\item Definir la ciencia y explicar el camino para su conocimiento, aprovechamiento. Definir conocimiento, aprendizaje, sus diferencias y caminos para su logro.
	\item Explicar lo que es la comprensi�n y ejercitarse en al expresi�n o transmisi�n del conocimiento comprendido.
\end{unitgoals}
\end{unit}

\begin{unit}{Tercera  Unidad}{Perez}{15}{1}
\begin{topics}
	\item Auto evaluaci�n de recursos y uso de m�todos.
	\item El subrayado y el resumen.
	\item La toma de apuntes.
	\item Los esquemas y cuadros sint�cticos.
	\item Los mapas conceptuales.
	\item Los mapas mentales.
	\item Las inteligencias m�ltiples.
	\item La conducta asertiva.
	\item La resiliencia y el reto acad�mico.
	\item El estilo personal de trabajo.
\end{topics}
\begin{unitgoals}
	\item Evidenciar conciencia y dominio aceptable de las principales t�cnicas de estudio, explicar y formular estrategias para las necesidades acad�micas.
	\item Definir el estilo personal de trabajo, en concordancia con sus caracter�sticas personales para el trabajo intelectual.
\end{unitgoals}
\end{unit}

\begin{unit}{Cuarta Unidad}{Blythe,Buzan,Goleman,FloresA}{15}{1}
\begin{topics}
	\item El estilo personal de estudio.
	\item Elecci�n de recursos.
	\item La lectura.
	\item Los ex�menes.
	\item El trabajo grupal y los roles en los grupos de trabajo.
	\item El trabajo de redacci�n.
	\item El aparto cr�tico.
	\item El articulo publicable.
\end{topics}
\begin{unitgoals}
	\item Aplicar adecuadamente las t�cnicas mas apropiadas a las necesidades del estudio, a la naturaleza del curso y a las exigencias del compromiso acad�mico.
	\item Aplicar t�cnicas y recursos adecuados para el estudio grupal, los ex�menes y la preparaci�n de trabajos.
\end{unitgoals}
\end{unit}

\begin{coursebibliography}
\bibfile{GeneralEducation/FG101}
\end{coursebibliography}

\end{syllabus}
