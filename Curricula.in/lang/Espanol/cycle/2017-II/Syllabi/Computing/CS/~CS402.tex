\begin{syllabus}

\course{CS402. Proyecto II}{Obligatorio}{CS402}

\begin{justification}
Este curso tiene por objetivo que el alumno pueda realizar un estudio del estado del arte apropiado al tema que ha escogido para su tesis.
\end{justification}

\begin{goals}
\item Que el alumno realice un estudio del estado del arte de un �rea espec�fica de investigaci�n en computaci�n y que muestre dominio en el tema desde el origen de esa l�nea de investigaci�n.
\item Los entregables de este curso son:
	\begin{description}
		\item [Avance parcial:] Bibliograf�a s�lida y avance de un Reporte T�cnico.
		\item [Final:] Reporte T�cnico conclu�do y experimentos comparativos entre dos o tres t�cnicas.
	\end{description}
\end{goals}

\begin{outcomes}
\ExpandOutcome{a}{3}
\ExpandOutcome{b}{4}
\ExpandOutcome{c}{3}
\ExpandOutcome{e}{5}
\ExpandOutcome{f}{3}
\ExpandOutcome{h}{4}
\ExpandOutcome{i}{5}
\ExpandOutcome{l}{3}
\end{outcomes}

\begin{unit}{Levantamiento del estado del arte}{ieee,acm,citeseer}{60}{1}
  \begin{topics}
      \item Realizar un estudio profundo del estado del arte en un determinado t�pico del �rea de Computaci�n.
      \item Redacci�n de art�culos t�cnicos en computaci�n.
  \end{topics}
  \begin{unitgoals}
      \item Hacer un levantamiento bibliogr�fico del estado del arte del tema escogido (esto significa muy probablemente 1 o 2 cap�tulos de marco te�rico adem�s de la introducci�n que es el cap�tulo I de la tesis)
      \item Redactar un documento en latex con mayor calidad que en Proyecto I (dominar tablas, figuras, ecuaciones, �ndices, bibtex, referencias cruzadas, citaciones, pstricks)
      \item Tratar de hacer las presentaciones utilizando prosper
      \item Mostrar experimentos b�sicos
   \end{unitgoals}
\end{unit}
\begin{coursebibliography}
\bibfile{Computing/CS/CS401}

\end{coursebibliography}
\end{syllabus}
