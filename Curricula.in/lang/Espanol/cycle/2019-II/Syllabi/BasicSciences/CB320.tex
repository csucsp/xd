
\begin{syllabus}

\course{CB320. Ciencia de los Materiales}{Obligatorio}{CB320}

\begin{justification}
La introducci�n y la innovaci�n de este curso empieza con la presentaci�n selecta de los fundamentos generales sobre Ciencia de los materiales e Ingenier�a.
Luego, se enfoca en seminarios sobre la familia de materiales: metales y aleaciones, cer�micos y vidrios, pol�meros y copol�meros, y compuestos y nanomateriales.
Las aplicaciones abarcan materiales tradicionales y de vanguardia. EL estudido de estas aplicaciones cubre el papel desempe�ado por los materiales, 
las mismas aplicaciones y su relevancia. Casos avanzados sobre materiales e innovadores aplicaciones de relevancia potencial sobre el contexto peruano son cubiertos.

% This introductory and innovative course starts by presenting selected overall
% fundamentals of Materials Science and Engineering. Next, faculty and student seminars
% focus on specific families of materials: metals \& alloys, ceramics \& glasses, polymers \&
% copolymers, and composites \& nanomaterials. Applications encompass both traditional
% and cutting-edge uses of materials. The study of these applications covers the role played
% by the materials, the applications themselves and their relevance. Cases of breakthrough
% materials and innovative applications of potential relevance to the Peruvian context are
% covered (be they already employed in the country or not).

\end{justification}

\begin{goals}
\item Capacity for teamwork.
\item Capacity to identify Engineering problems.
\item Capacity to communicate orally.
\item Capacity to communicate in writing.

%item Capacity for teamwork.
%item Capacity to identify Engineering problems.
%item Capacity to communicate orally.
%item Capacity to communicate in writing.
\end{goals}

\begin{outcomes}
\ShowOutcome{a}{3}
\ShowOutcome{i}{3}
\ShowOutcome{j}{4}

% Specific outcomes : 
% - Understanding basic properties and how they define the overall behavior of materials;
% - Knowing how basic properties of materials can be tailored;
% - Discovering processing and handling methods proper to the main families of materials;
% - Realizing the economic, environmental, health and safety impacts associated with materials;
% 
% - Acknowledging the importance of acquiring a basic understanding of materials for progressing autonomously in the area.
% Transversal outcomes:
% - Communicating orally with effectiveness using specific technical vocabulary ;
% - Getting trained in working in groups.

\end{outcomes}


\begin{unit}{Applied understanding of materials}{MSaE2014}{0}{3}
%begin{unit}{Applied understanding of materials}{MSaE2014}{0}{3}
\begin{topics}
      \item . %Course presentation and organization
      \item . %Importance of materials for Engineering Sciences
      \item . %Overall classification of materials
      \item . %Desirable functions for materials
	  \begin{subtopics}
	  \item . %Mechanical properties (e.g. structural materials)
	  \item . %Electrical and heat conductivity (e.g. circuits, cells, sensors)
	  \item . %Chemical resistance (e.g. chemical compatibility; corrosion)
	  \item . %Environmental and biological compatibility
	  \end{subtopics}
      \item . %Overall fundamentals
	  \begin{subtopics}
	  \item . %Chemical bond and its impact on malleability and ductility
	  \item . %Alloys and phase diagrams
	  \item . %Crystal growth and defects
	  \item . %Chemical reactivity (defects, grain boundaries)
	  \item . %Galvanic pairs
	  \item . %Pourbaix diagrams
	  \item . %Band theory and heat and electrical conduction
	  \item . %Conductors, semiconductors
	  \end{subtopics}      
\end{topics}
   \begin{learningoutcomes}
    \item . %Understanding the overall fundamentals and desireble functions for materials.
    \item . %Acknowledging the importance of acquiring a basic understanding of materials for progressing autonomously in the area.
   \end{learningoutcomes}
\end{unit}


\begin{unit}{Dealing with Metals \& Alloys}{MSaE2014}{0}{3}
%begin{unit}{Dealing with Metals \& Alloys}{MSaE2014}{0}{3}
\begin{topics}
      \item . %Other specific fundamentals needed
      \item . %Properties and correlated applications
      \item . %Survey of metals \& alloys - traditional applications
      \item . %Survey of metals \& alloys - cutting-edge applications
\end{topics}
   \begin{learningoutcomes}
      \item . %Recognize the purpose, requirements, and general characteristics of Metals and Alloys.
   \end{learningoutcomes}
\end{unit}


\begin{unit}{Dealing with Ceramics \& Glasses}{MSaE2014}{0}{3}
%begin{unit}{Dealing with Ceramics \& Glasses}{MSaE2014}{0}{3}
\begin{topics}
      \item . %Other specific fundamentals needed
      \item . %Properties and correlated applications
      \item . %Survey of ceramics \& glasses - traditional applications
      \item . %Survey of ceramics \& glasses - cutting-edge applications
\end{topics}
   \begin{learningoutcomes}

      \item . %Recognize the purpose, requirements, and general characteristics of Ceramics and Glasses.
   \end{learningoutcomes}
\end{unit}

\begin{unit}{Dealing with Polymers \& Copolymers}{MSaE2014}{0}{3}
% begin{unit}{Dealing with Polymers \& Copolymers}{MSaE2014}{0}{3}
\begin{topics}
      \item . %Other specific fundamentals needed
      \item . %Properties and correlated applications
      \item . %Survey of polymers \& copolymers - traditional applications
      \item . %Survey of polymers \& copolymers - cutting-edge applications
\end{topics}
   \begin{learningoutcomes}

      \item . %Recognize the purpose, requirements, and general characteristics of Polymers and Copolymers.
   \end{learningoutcomes}
\end{unit}

\begin{unit}{Dealing with Composites and with Nanomaterials}{MSaE2014}{0}{3}
% begin{unit}{Dealing with Composites and with Nanomaterials}{MSaE2014}{0}{3}
\begin{topics}
      \item . %Other specific fundamentals needed
      \item . %Properties and correlated applications
      \item . %Survey of composites- traditional and cutting-edge applications
      \item . %Survey of nanomaterials- traditional and cutting-edge applications
\end{topics}
   \begin{learningoutcomes}

      \item . %Recognize the purpose, requirements, and general characteristics of Composites and Nanomaterials.
   \end{learningoutcomes}
\end{unit}

\begin{unit}{Searching new materials and developing applications}{MSaE2014}{0}{3}
%begin{unit}{Searching new materials and developing applications}{MSaE2014}{0}{3}
\begin{topics}
      \item . %Innovative pair "material - application", e.g.:
	  \begin{subtopics}
	  \item . % Art and archeological conservation/restoration
	  \item . % Environment
	  \item . % Nanomaterials
	  \item . % Bioengineering
	  \item . % 3D-printing	
	  \item . % Functional materials
	  \item . % Packaging
	  \end{subtopics}  
\end{topics}
   \begin{learningoutcomes}

      \item . %Ability to integrate understanding of the new materials to developing applications
   \end{learningoutcomes}
\end{unit}


\begin{coursebibliography}
\bibfile{BasicSciences/CB320}
\end{coursebibliography}

\end{syllabus}

