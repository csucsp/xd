\begin{syllabus}

\course{FG202. Apreciaci�n Literaria}{Electivos}{FG202}

\begin{justification}
La Universidad Cat�lica San Pablo dentro de su proyecto educativo se�ala la importancia de la formaci�n humana de sus alumnos, que mejor veh�culo para contribuir con este objetivo que la Literatura que es un importante medio de expresi�n humana, a trav�s de esta conocemos el alma de los pueblos y el pensamiento, vivencias, sue�os, sufrimientos y esperanzas del hombre a trav�s de los tiempos.
\end{justification}

\begin{goals}
\item Este curso contribuye a entender la literatura como un medio de expresi�n del ser humano.
\item Desarrollar su sensibilidad para apreciar la funci�n est�tica del lenguaje.
\end{goals}

\begin{outcomes}
    \item \ShowOutcome{f}{2}
    \item \ShowOutcome{�}{2}
\end{outcomes}
\begin{competences}
    \item \ShowCompetence{C24}{�, f}
\end{competences}

\begin{unit}{La Literatura}{La Literatura}{C�ceres, Bello}{9}{C24}
\begin{topics}
	\item La comunicaci�n literaria.
	\item G�neros literarios
	\item El comentario y an�lisis de textos.
	\item El lenguaje, herramienta fundamental de la literatura: Sus posibilidades e imposibilidades 
\end{topics}
\begin{learningoutcomes}
	\item Definir y caracterizar la Literatura, indicando g�neros, recursos y lenguaje [\Usage].
	\item Valorar la expresi�n Literatura en su  esencia, adoptando una postura de apertura y sensibilidad hacia ella [\Usage].
\end{learningoutcomes}
\end{unit}

\begin{unit}{El clasicismo}{El clasicismo}{Torres, Homero, Sanzos, Alighieri}{15}{C24}
\begin{topics}
	\item Homero ``La Iliada''
	\item S�focles ``Edipo Rey''
	\item Virgilio ``La Eneida''
	\item Literatura Cristiana ``La Biblia''
\end{topics}
\begin{learningoutcomes}
	\item Conocer  y valorar el Clasi-cismo y la Literatura Cl�sica. [\Familiarity].
	\item eer, comentar y apreciar fragmentos selectos de Literatura Cl�sica. [\Usage].
\end{learningoutcomes}
\end{unit}

\begin{unit}{Literatura Medieval}{Literatura Medieval}{Hugo, Hemingway, Goethe}{18}{C24}
\begin{topics}
	\item Dante Alighieri ``La Divina Comedia''
	\item San Agust�n ``Confesiones?
	\item El Renacimiento
	\item Shakespeare ``Hamlet''
	\item Miguel  de Cervantes ``Don Quijote''
\end{topics}
\begin{learningoutcomes}
	\item Se�alar caracter�sticas b�sicas de la Edad Media y de la Literatura Medieval [\Usage].
	\item Leer, analizar y valorar textos de literatura medieval.[\Usage].
	\item Caracterizar el Renacimiento. [\Usage].
	\item Valorar el Renacimiento. Valorar textos renacentistas. [\Usage].
\end{learningoutcomes}
\end{unit}

\begin{unit}{Cuarta Unidad}{Cuarta Unidad}{Hugo, Hemingway, Goethe}{18}{C24}
\begin{topics}
	\item Goethe ``Werther''
	\item Allan Poe ``Narraciones Extraordinarias''
	\item A. B�cquer ``Rimas y Leyendas''
	\item Mariano Melgar ``Yarav�''
	\item Realismo Ruso, Fedor Dostoievsky ``Crimen y Castigo''
	\item Manuel Gonz�les Prada ``P�ginas Libres''
\end{topics}
\begin{learningoutcomes}
	\item Valorar la literatura del Romanticismo a trav�s de la lectura de textos rom�nticos.
	\item Caracterizar el Realismo, leer y analizar fragmentos de literatura realista[\Usage].
\end{learningoutcomes}
\end{unit}

\begin{unit}{El Modernismo}{El Modernismo}{Hugo, Hemingway, Goethe}{18}{C24}
\begin{topics}
	\item Rub�n Dar�o
	\item Antonio Machado
	\item El postmodernismo: Gabriela Mistral, C�sar Vallejo
\end{topics}
\begin{learningoutcomes}
	\item Caracterizar el Movimiento Literario Modernista y Post Modernista. [\Usage].
	\item Leer y Valorar selectos textos modernistas y Post Modernistas. [\Usage].
\end{learningoutcomes}
\end{unit}



\begin{coursebibliography}
\bibfile{GeneralEducation/FG101}
\end{coursebibliography}
\end{syllabus}
