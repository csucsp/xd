\begin{syllabus}

\course{CS3P1. Computaci�n Paralela y Distribu�da}{Obligatorio}{CS3P1}

\begin{justification}
La �ltima d�cada ha tra�do un crecimiento explosivo en computaci�n con multiprocesadores, incluyendo 
los procesadores de varios n�cleos y centros de datos distribuidos. Como resultado, la computaci�n 
paralela y distribuida se ha convertido de ser un tema ampliamente electivo para ser uno de los principales componentes
en la malla estudios en ciencia de la computaci�n de pregrado. Tanto la computaci�n paralela como la distribuida implica 
la ejecuci�n simult�nea de m�ltiples procesos, cuyas operaciones tienen el potencial para 
intercalar de manera compleja. La computaci�n paralela y distribuida construye sobre cimientos en muchas 
�reas, incluyendo la comprensi�n de los conceptos fundamentales de los sistemas, tales como: concurrencia 
y ejecuci�n en paralelo, consistencia en el estado/manipulaci�n de la memoria, y latencia. La 
comunicaci�n y la coordinaci�n entre los procesos tiene sus cimientos en el paso de mensajes y modelos de 
memoria compartida de la computaci�n y conceptos algor�tmicos como atomicidad, el consenso y espera condicional. 
El logro de aceleraci�n en la pr�ctica requiere una comprensi�n de algoritmos paralelos, estrategias para la 
descomposici�n problema, arquitectura de sistemas, estrategias de implementaci�n y an�lisis de 
rendimiento. Los sistemas distribuidos destacan los problemas de la seguridad y tolerancia a 
fallos, hacen hincapi� en el mantenimiento del estado replicado e introducen problemas adicionales en el campo de 
las redes de computadoras.
\end{justification}

\begin{goals}
\item Que el alumno sea capaz de crear aplicaciones paralelas de mediana complejidad aprovechando eficientemente m�quinas con m�ltiples n�cleos.
\item Que el alumno sea capaz de comparar aplicaciones secuenciales y paralelas.
\item Que el alumno sea capaz de convertir, cuando la situaci�n lo amerite, aplicaciones secuenciales a paralelas de forma eficiente.
\end{goals}

\begin{outcomes}
    \item \ShowOutcome{a}{2} 
    \item \ShowOutcome{b}{2} % Analizar problemas e identificar y definir los requerimientos computacionales apropiados para su soluci�n
    \item \ShowOutcome{i}{2} % Utilizar t�cnicas y herramientas actuales necesarias para la pr�ctica de la computaci�n
    \item \ShowOutcome{j}{2} % Aplicar la base matem�tica, principios de algoritmos y la teor�a de la Ciencia de la Computaci�n en el modelamiento y dise�o de sistemas computacionales de tal manera que demuestre comprensi�n de los puntos de equilibrio involucrados en la opci�n escogida.
\end{outcomes}

\begin{competences}
    \item \ShowCompetence{C2}{a} % Capacidad para tener una perspectiva cr�tica y creativa para identificar y resolver problemas utilizando el pensamiento computacional
    \item \ShowCompetence{C4}{b}
    \item \ShowCompetence{C16}{i} % Capacidad para identificar temas avanzados de computaci�n y de la comprensi�n de las fronteras de la disciplina. 
    \item \ShowCompetence{CS2}{i} % Identificar y analizar los criterios y especificaciones apropiadas a los problemas espec�ficos, y planificar estrategias para su soluci�n.
    \item \ShowCompetence{CS3}{j} % Analizar el grado en que un sistema basado en el ordenador cumple con los criterios definidos para su uso actual y futuro desarrollo.
  \item \ShowCompetence{CS6}{j} % Evaluar los sistemas en t�rminos de atributos de calidad en general y las posibles ventajas y desventajas que se presentan en el problema dado.
\end{competences}

\begin{unit}{\PDParallelismFundamentals}{}{peterpacheco,matloff,quinn}{18}{C2}
\begin{topics}%
    \item \PDParallelismFundamentalsTopicMultiple
    \item \PDParallelismFundamentalsTopicGoals
    \item \PDParallelismFundamentalsTopicParallelism
    \item \PDParallelismFundamentalsTopicProgramming
\end{topics}    
\begin{learningoutcomes}%
    \item \PDParallelismFundamentalsLODistinguishUsing~[\Familiarity] %
    \item \PDParallelismFundamentalsLODistinguishMultiple~[\Familiarity] %
    \item \PDParallelismFundamentalsLODistinguishData~[\Familiarity] %
\end{learningoutcomes}%
\end{unit}

\begin{unit}{\PDParallelArchitecture}{}{peterpacheco,wenmei,sanders}{12}{C4}
\begin{topics}%
    \item \PDParallelArchitectureTopicMulticore
    \item \PDParallelArchitectureTopicShared
    \item \PDParallelArchitectureTopicSymmetric
    \item \PDParallelArchitectureTopicSimd
    \item \PDParallelArchitectureTopicGpu
    \item \PDParallelArchitectureTopicFlynns
    \item \PDParallelArchitectureTopicInstruction
    \item \PDParallelArchitectureTopicMemory
    \item \PDParallelArchitectureTopicTopologies
\end{topics}
\begin{learningoutcomes}%
    \item \PDParallelArchitectureLOExplainTheShared~[\Assessment] %
    \item \PDParallelArchitectureLODescribeTheAndKey~[\Assessment] %
    \item \PDParallelArchitectureLOCharacterizeTheTasks~[\Usage] %
    \item \PDParallelArchitectureLODescribeTheLimitationsVs~[\Usage] %
    \item \PDParallelArchitectureLOExplainTheEach~[\Usage] %
    \item \PDParallelArchitectureLODescribeTheMaintaining~[\Familiarity] %
    \item \PDParallelArchitectureLODescribeTheChallenges~[\Familiarity] %
\end{learningoutcomes}%
\end{unit}

\begin{unit}{\PDParallelDecomposition}{}{peterpacheco,matloff,quinn}{18}{C16}
\begin{topics}%
    \item \PDParallelDecompositionTopicNeed
    \item \PDParallelDecompositionTopicIndependence
    \item \PDParallelDecompositionTopicBasic
    \item \PDParallelDecompositionTopicTask
    \item \PDParallelDecompositionTopicData
    \item \PDParallelDecompositionTopicActors
\end{topics}
\begin{learningoutcomes}%
    \item \PDParallelDecompositionLOExplainWhyNecessary~[\Usage] %
    \item \PDParallelDecompositionLOIdentifyOpportunities~[\Familiarity] %
    \item \PDParallelDecompositionLOWriteAScalable~[\Usage] %
    \item \PDParallelDecompositionLOParallelize~[\Usage] %
    \item \PDParallelDecompositionLOParallelizeAn~[\Usage] %
    \item \PDParallelDecompositionLOWriteAActors~[\Usage] %
\end{learningoutcomes}%
\end{unit}

\begin{unit}{\PDCommunicationandCoordination}{}{peterpacheco,matloff,quinn}{18}{C16}
\begin{topics}%
    \item \PDCommunicationandCoordinationTopicShared
    \item \PDCommunicationandCoordinationTopicConsistency
    \item \PDCommunicationandCoordinationTopicMessage
    \item \PDCommunicationandCoordinationTopicAtomicity
    \item \PDCommunicationandCoordinationTopicConsensus
    \item \PDCommunicationandCoordinationTopicConditional
\end{topics}
\begin{learningoutcomes}%
    \item \PDCommunicationandCoordinationLOUseMutual~[\Usage] %
    \item \PDCommunicationandCoordinationLOGiveAn~[\Familiarity] %
    \item \PDCommunicationandCoordinationLOGiveAnA~[\Usage] %
    \item \PDCommunicationandCoordinationLOExplainWhenMulticast~[\Familiarity] %
    \item \PDCommunicationandCoordinationLOWriteACorrectly~[\Usage] %
    \item \PDCommunicationandCoordinationLOGiveAnAWhich~[\Familiarity] %
    \item \PDCommunicationandCoordinationLOUseSemaphores~[\Usage] %
\end{learningoutcomes}%
\end{unit}

\begin{unit}{\PDParallelAlgorithmsAnalysisandProgramming}{}{matloff,quinn}{18}{CS2}
\begin{topics}%
    \item \PDParallelAlgorithmsAnalysisandProgrammingTopicCritical
    \item \PDParallelAlgorithmsAnalysisandProgrammingTopicSpeed
    \item \PDParallelAlgorithmsAnalysisandProgrammingTopicNaturally
    \item \PDParallelAlgorithmsAnalysisandProgrammingTopicParallel
    \item \PDParallelAlgorithmsAnalysisandProgrammingTopicParallelGraph
    \item \PDParallelAlgorithmsAnalysisandProgrammingTopicParallelMatrix
    \item \PDParallelAlgorithmsAnalysisandProgrammingTopicProducer
    \item \PDParallelAlgorithmsAnalysisandProgrammingTopicExamples
\end{topics}
\begin{learningoutcomes}%
        \item \PDParallelAlgorithmsAnalysisandProgrammingLODefineCritical~[\Familiarity] %
        \item \PDParallelAlgorithmsAnalysisandProgrammingLOComputeTheSpan~[\Usage] %
        \item \PDParallelAlgorithmsAnalysisandProgrammingLODefineSpeed~[\Familiarity] %
        \item \PDParallelAlgorithmsAnalysisandProgrammingLOIdentifyIndependent~[\Usage] %
        \item \PDParallelAlgorithmsAnalysisandProgrammingLOCharacterizeFeatures~[\Familiarity] %
        \item \PDParallelAlgorithmsAnalysisandProgrammingLOImplementAAnd~[\Usage] %
        \item \PDParallelAlgorithmsAnalysisandProgrammingLODecompose~[\Usage] %
        \item \PDParallelAlgorithmsAnalysisandProgrammingLOProvideAn~[\Usage] %
        \item \PDParallelAlgorithmsAnalysisandProgrammingLOGiveExamplesWhere~[\Usage] %
        \item \PDParallelAlgorithmsAnalysisandProgrammingLOImplementAAlgorithm~[\Usage] %
        \item \PDParallelAlgorithmsAnalysisandProgrammingLOIdentifyIssuesIn~[\Usage] %
\end{learningoutcomes}%
\end{unit}

\begin{unit}{\PDParallelPerformance}{}{peterpacheco,matloff,wenmei,sanders}{18}{CS3}
\begin{topics}%
    \item \PDParallelPerformanceTopicLoad
    \item \PDParallelPerformanceTopicPerformance
    \item \PDParallelPerformanceTopicScheduling
    \item \PDParallelPerformanceTopicEvaluating
    \item \PDParallelPerformanceTopicData
    \item \PDParallelPerformanceTopicPower
\end{topics}
\begin{learningoutcomes}%
    \item \PDParallelPerformanceLODetect~[\Usage] %
    \item \PDParallelPerformanceLOCalculateThe~[\Usage] %
    \item \PDParallelPerformanceLODescribeHowLayout~[\Familiarity] %
    \item \PDParallelPerformanceLODetectAnd~[\Usage] %
    \item \PDParallelPerformanceLOExplainTheScheduling~[\Familiarity] %
    \item \PDParallelPerformanceLOExplainPerformance~[\Familiarity] %
    \item \PDParallelPerformanceLOExplainTheTrade~[\Familiarity] %
\end{learningoutcomes}%
\end{unit}



\begin{coursebibliography}
\bibfile{Computing/CS/CS3P1}
\end{coursebibliography}

\end{syllabus}
