\begin{syllabus}

\course{CS103O. Algoritmos y Estructuras de Datos}{Obligatorio}{CS103O}

\begin{justification}
El fundamento te�rico de todas las ramas de la inform�tica descansa sobre los algoritmos y estructuras de datos, este curso brindar� a los participantes una introducci�n a estos t�mas, formando as� una base que servir� para los siguientes cursos en la carrera.
\end{justification}

\begin{goals}
\item Hacer que el alumno entienda la importancia de los algoritmos para la soluci�n de problemas.
\item Introducir al alumno hacia el campo de la aplicaci�n de las estructuras de datos.
\end{goals}

\begin{outcomes}
\ExpandOutcome{a}{4}
\ExpandOutcome{b}{4}
\ExpandOutcome{c}{3}
\ExpandOutcome{d}{3}
\ExpandOutcome{i}{4}
\ExpandOutcome{j}{4}
\ExpandOutcome{k}{3}
\end{outcomes}

\begin{unit}{\PFFundamentalDataStructuresDef}{Cormen2009,Fager2014}{8}{5}
   \PFFundamentalDataStructuresAllTopics
   \PFFundamentalDataStructuresAllObjectives
\end{unit}

\begin{unit}{\PFRecursionDef}{Cormen2009,Fager2014}{4}{5}
    \PFRecursionAllTopics
    \PFRecursionAllObjectives
\end{unit}

\begin{unit}{\ALFundamentalAlgorithmsDef}{Cormen2009,Fager2014}{12}{4}
    \ALFundamentalAlgorithmsAllTopics
    \ALFundamentalAlgorithmsAllObjectives
\end{unit}

\begin{unit}{Grafos}{Cormen2009,Fager2014}{12}{5}
   \begin{topics}
    \item Concepto de Grafos.
    \item Grafos Dirigidos y Grafos no Dirigidos.
    \item Utilizaci�n de los Grafos.
    \item Medida de la Eficiencia. En tiempo y espacio.
    \item Matrices de Adyacencia.
    \item Matrices de Adyacencia etiquetada.
    \item Listas de Adyacencia.
    \item Implementaci�n de Grafos usando Matrices de Adyacencia.
    \item Implementaci�n de Grafos usando Listas de Adyacencia.
    \item Inserci�n, B�squeda y Eliminaci�n de nodos y aristas.
    \item Algoritmos de b�squeda en grafos.
   \end{topics}
   \begin{unitgoals}
      \item  Adquirir destreza para realizar una implementaci�n correcta.
      \item  Desarrollar los conocimientos para decidir cuando es mejor usar una t�cnica de implementaci�n que otra.
   \end{unitgoals}
\end{unit}

\begin{unit}{Matrices Esparzas}{Cormen2009,Fager2014}{8}{5}
   \begin{topics}
    \item  Conceptos  Iniciales.
    \item  Matrices poco densas
    \item  Medida de la Eficiencia en Tiempo  y en Espacio
    \item  Creaci�n de la matriz esparza est�tica vs Din�micas.
    \item  M�todos de inserci�n, b�squeda y eliminaci�n
   \end{topics}

\begin{unitgoals}
      \item Comprender el uso y implementacion de matrices esparzas.
   \end{unitgoals}
\end{unit}

\begin{unit}{Arboles Equilibrados}{Cormen2009,Fager2014}{16}{5}
   \begin{topics}
        \item �rboles AVL.
	\item Medida de la Eficiencia.
	\item Rotaciones Simples y Compuestas
	\item Inserci�n, Eliminaci�n y B�squeda.
	\item �rboles B , B+ B* y Patricia.
   \end{topics}

   \begin{unitgoals}
      \item Comprender las funciones b�sicas de estas estructuras complejas con el fin de adquirir la capacidad para su implementaci�n.
   \end{unitgoals}
\end{unit}



\begin{coursebibliography}
\bibfile{Computing/CS/CS103O}
\end{coursebibliography}

\end{syllabus}
