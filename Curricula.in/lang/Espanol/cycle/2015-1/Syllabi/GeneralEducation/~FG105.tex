\begin{syllabus}

\course{FG105. Apreciaci�n de la M�sica}{Electivo}{FG105}

\begin{justification}
El egresado de nuestra universidad, no s�lo deber� ser un excelente profesional, conocedor de la m�s avanzada tecnolog�a, sino tambi�n, un ser humano sensible y de amplia cultura. En esta perspectiva, el curso de apreciaci�n musical, permite al alumno y alumna, conocer, entender y valorar un arte creativo y humano.
\end{justification}

\begin{goals}
\item Formaci�n Humana: Valorar la belleza y el potencial creador del ser humano.
\item Formaci�n Profesional: Conocer los principales elementos constitutivos de la obra musical, como tambi�n los principales hechos hist�ricos del desarrollo de la m�sica, para incrementar su bagaje cultural y apreciar la m�sica con objetividad y criticidad.
\end{goals}

\begin{outcomes}
\ExpandOutcome{FH}{1}
\ExpandOutcome{HU}{1}
\end{outcomes}

\begin{unit}{El lenguaje de la m�sica}{Lovon,Salvat,Vives,Penaloza2009}{11}{1}
\begin{topics}
	\item La m�sica en la vida del hombre. Concepto. El Sonido: cualidades.
	\item Los elementos de la m�sica. Actividades y audiciones.
\end{topics}
\begin{unitgoals}
	\item Dotar al alumno de un lenguaje musical b�sico, que le permita apreciar y emitir un juicio con propiedad.
\end{unitgoals}
\end{unit}

\begin{unit}{El quehacer de la m�sica}{Lovon,Salvat,Vives,Suarez2005Camino}{11}{1}
\begin{topics}
	\item La voz, el canto y sus int�rpretes.
	\item Los instrumentos musicales. El conjunto instrumental.
	\item El estilo, g�nero y las formas musicales.
	\item Actividades y audiciones.
\end{topics}
\begin{unitgoals}
	\item Que el alumno conozca, discrimine y aprecie los elementos que integran la obra de arte musical.
\end{unitgoals}
\end{unit}

\begin{unit}{Historia de la m�sica}{Lovon,Salvat,Vives,Schwanitz2002}{11}{1}
\begin{topics}
	\item El origen de la m�sica - fuentes. La m�sica en la antig�edad.
	\item La m�sica medieval: M�sica religiosa.  Canto Gregoriano. M�sica profana.
	\item El Renacimiento: M�sica instrumental y m�sica vocal.
	\item El Barroco y sus representantes. Nuevos instrumentos, nuevas formas.
	\item El Clasicismo. Las formas cl�sicas y sus m�s destacados representantes.
	\item El Romanticismo y el Nacionalismo, caracter�sticas generales instrumentos y formas. Las escuelas nacionalistas europeas.
	\item La m�sica contempor�nea: Impresionismo y las nuevas corrientes de vanguardia.
\end{topics}
\begin{unitgoals}
	\item Que el alumno conozca y distinga con precisi�n los diferentes momentos del desarrollo musical.
	\item Dotar al alumno de un repertorio m�nimo que le permita poner en pr�ctica lo aprendido antes de emitir una apreciaci�n cr�tica de ellas.
\end{unitgoals}
\end{unit}

\begin{unit}{M�sica popular}{Lovon,Salvat,Vives}{12}{1}
\begin{topics}
	\item Principales corrientes musicales del Siglo XX.
	\item La m�sica peruana: Aut�ctona, Mestiza, Manifestaciones musicales actuales.
	\item M�sica arequipe�a, principales expresiones.
	\item M�sica latinoamericana y sus principales manifestaciones.
\end{topics}
\begin{unitgoals}
	\item Que el alumno conozca e identifique las diferentes manifestaciones populares actuales. 
	\item Que el alumno Se identifique con sus ra�ces musicales.
\end{unitgoals}
\end{unit}

\begin{coursebibliography}
\bibfile{GeneralEducation/FG105}
\end{coursebibliography}

\end{syllabus}
