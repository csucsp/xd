\begin{syllabus}

\course{CS411. Pasant�a II}{Obligatorio}{CS411}

\begin{justification}
Se constituye en un espacio de comprobaci�n y validaci�n de conocimientos te�ricos-pr�cticos desde su formaci�n profesional, permite que el pasante se relacione con la utilizaci�n y avances de l�neas t�cnicas y tecnol�gicas en las que se apoya la empresa y/o instituci�n para su operaci�n, hace m�rito de su capacidad formativa para identificar carencias y capacidades en la empresa a trav�s de la investigaci�n. El pasante ser� participante o interviniente en los programas de demanda de las Ciencias de la Computaci�n que aplica o desea aplicar la empresa, como planteamiento deliberante, capaz de dar a conocer el nivel de formaci�n recibida y las proyecciones que visualiza en el plano tecnol�gico.  
\end{justification}

\begin{goals}
\item Potenciar los conocimientos te�ricos y pr�cticos de formaci�n profesional y 
canalizar los procesos t�cnicos de investigaci�n y propuestas en sistemas de la 
Ciencias de la Computaci�n. 
\end{goals}

\begin{outcomes}
\ExpandOutcome{f}{1}
\ExpandOutcome{h}{1}
\ExpandOutcome{n}{1}
\end{outcomes}

\begin{unit}{Pasantia I}{Pasantia}{75}{1}
   \begin{topics}
      \item Observaci�n, apoyo y verificaci�n de los procesos pr�cticos-t�cnicos que desarrolla la empresa en computaci�n.
   \end{topics}

   \begin{unitgoals}
      \item Que el alumno tenga una experiencia en el campo laboral que le permita consolidar los conocimientos adquiridos en su carrera.
   \end{unitgoals}
\end{unit}
   
\begin{coursebibliography}
\bibfile{Computing/CS/CS410}
\end{coursebibliography}
\end{syllabus}
