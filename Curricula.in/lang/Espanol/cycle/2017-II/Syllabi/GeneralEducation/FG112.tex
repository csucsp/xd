\begin{syllabus}

\course{FG112. Persona, Matrimonio y Familia}{Electivos}{FG112}

\begin{justification}
Los tiempos actuales muestran la necesidad - cada vez m�s apremiante-  de una adecuada visi�n antropol�gica sobre el matrimonio y la familia.

La referencia de la familia como instituci�n natural fundada en el matrimonio, viene en diversas organizaciones internacionales promovida como una construcci�n social y cultural que tiende a desconocer la complementariedad del var�n y la mujer.

Este curso intentar� mostrar los presupuestos de una perspectiva de familia que destaque la riqueza de la familia como aut�ntico eje de desarrollo humano.
\end{justification}

\begin{goals}
	\item Conocer la naturaleza de la persona humana y su dignidad, y comprender que la familia es una comunidad de vida y amor, fundada en el matrimonio entre un hombre y una mujer para toda la vida, en orden al perfeccionamiento mutuo y a la procreaci�n y educaci�n de los hijos.
\end{goals}

\begin{outcomes}
    \item \ShowOutcome{g}{2}
    \item \ShowOutcome{�}{2}
    \item \ShowOutcome{o}{2}
\end{outcomes}
\begin{competences}
    \item \ShowCompetence{C10}{g, �, o}
    \item \ShowCompetence{C20}{g}
\end{competences}

\begin{unit}{La persona humana}{La persona humana}{caffarra, d2006filosofia}{12}{C10, C20}
\begin{topics}
	\item Personal humana
	\item Integraci�n de la persona humana.
	\item Emociones y afectos del ser.
	\item La persona: ser integral (seres sexuados).
	\item Dignidad humana: Conceptos, Elementos, Caracter�sticas.
\end{topics}

\begin{learningoutcomes}
	\item Comprender los fundamentos que permitan conocer a la persona valorando su dignidad [\Usage].
\end{learningoutcomes}
\end{unit}

\begin{unit}{Matrimonio: Aportes para una reflexi�n actual}{Matrimonio: Aportes para una reflexi�n actual}{PedroJuan, Hervada, QuesadaG, PedroJuan2, olaso2003etica, Ariza, Francesco}{12}{C20}
\begin{topics}
	\item Afectividad: Definici�n, importancia, caracter�sticas
    	\item El enamoramiento y el Amor: Definici�n e importancia, caracter�sticas 
	\item El matrimonio: El amor conyugal, qu� es y qu� no es matrimonio, finalidad- bienes, el matrimonio como fundamento de la familia	
\end{topics}
\begin{learningoutcomes}
	\item Comprender que el ser humano ha sido creado por amor y para el amor, que lo direcciona hacia una uni�n de las naturalezas (complementariedad) y como vocaci�n al matrimonio [\Usage].
\end{learningoutcomes}
\end{unit}

\begin{unit}{Retos y desaf�os en la familia en el siglo XX}{Retos y desaf�os en la familia en el siglo XX}{elosegui, la1996sexualidad, d2006filosofia}{15}{C10, C20}
\begin{topics}
	\item La Familia: La familia como principio antropol�gico. Subjetividad social de la familia
	\item Diagn�stico de la Familia: En cuanto a la durabilidad del matrimonio: convivencia y divorcio. En cuanto a la identidad del matrimonio: Ideolog�a de g�nero. En cuanto a la corresponsabilidad y deberes de la familia: violencia familiar.
	\item Desaf�o: Promoci�n de la Familia- Familia como eje irremplazable del desarrollo humano
\end{topics}
\begin{learningoutcomes}
	\item Comprender la importancia de la familia como c�lula fundamental de la sociedad y coraz�n de la civilizaci�n [\Usage].
\end{learningoutcomes}
\end{unit}

\begin{unit}{Promoci�n de la Familia}{Promoci�n de la Familia}{elosegui, la1996sexualidad, d2006filosofia}{6}{C20}
\begin{topics}
	\item Desaf�o: Promoci�n de la Familia. Familia como eje irremplazable del desarrollo humano. Poblaci�n y control demogr�fico. Los principios de Princeton 
	\item Pol�ticas Familiares: Pol�ticas sociales y familiares. Protecci�n del trabajo y seguridad social
\end{topics}
\begin{learningoutcomes}
	\item Comprender la importancia de la familia como c�lula fundamental de la sociedad y coraz�n de la civilizaci�n [\Usage].
\end{learningoutcomes}
\end{unit}



\begin{coursebibliography}
\bibfile{GeneralEducation/FG112}
\end{coursebibliography}

\end{syllabus}
