\begin{syllabus}

\course{CS255. Computaci�n Gr�fica}{Obligatorio}{CS255}

\begin{justification}
Ofrece una introducci�n para el �rea de Computaci�n Gr�fica, la
cual es una parte importante dentro de Ciencias de la Computaci�n.
El proposito de este curso es investigar los principios, t�cnicas
y herramientas fundamentales para esta �rea.
\end{justification}

\begin{goals}
\item Acercar al alumno a conceptos y t�cnicas usados en aplicaciones gr�ficas 3-D complejas.
\item Dar al alumno las herramientas necesarias para determinar que software gr�fico y que plataforma son los m�s adecuados para desarrollar una aplicaci�n espec�fica.
\end{goals}

\begin{outcomes}
\ExpandOutcome{a}{1}
\ExpandOutcome{b}{1}
\ExpandOutcome{i}{1}
\ExpandOutcome{j}{1}
\end{outcomes}

\begin{unit}{\GVGraphicSystemsDef}{Foley90,HearnAndBaker94}{6}{1}
	\GVGraphicSystemsAllTopics
	\GVGraphicSystemsAllObjectives
\end{unit}

\begin{unit}{\GVFundamentalTechniquesDef}{Foley90,HearnAndBaker94}{12}{1}
	\GVFundamentalTechniquesAllTopics
	\GVFundamentalTechniquesAllObjectives
\end{unit}

\begin{unit}{\GVBasicRenderingDef}{Foley90,HearnAndBaker94}{18}{1}
	\GVBasicRenderingAllTopics
	\GVBasicRenderingAllObjectives
\end{unit}

\begin{unit}{\GVGeometricModelingDef}{Foley90,HearnAndBaker94}{9}{1}
	\GVGeometricModelingAllTopics
	\GVGeometricModelingAllObjectives
\end{unit}

\begin{coursebibliography}
\bibfile{Computing/CS/CS255}
\end{coursebibliography}

\end{syllabus}
