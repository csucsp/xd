\begin{syllabus}

\course{FG203. Oratoria y Expresi�n Personal}{Electivo}{FG203}

\begin{justification}
En la sociedad competitiva como la nuestra, se exige que la persona sea un comunicador eficaz y sepa utilizar sus potencialidades a fin de resolver problemas y enfrentar los desaf�os del mundo moderno dentro de la actividad laboral, intelectual y social. Tener el conocimiento no basta, lo importante es saber comunicarlo y en la medida que la persona sepa emplear sus facultades comunicativas, derivara en �xito o fracaso aquello que tenga que realizar en su desenvolvimiento personal y profesional. Por ello es necesario para lograr un buen decir, recurrir a conocimientos, estrategias y recursos, que debe tener todo orador, para llegar con claridad, precisi�n y convicci�n al interlocutor.
\end{justification}

\begin{goals}
\item Este curso contribuye a que el profesional aprenda a expresar sus ideas de la forma adecuada.
\end{goals}

\begin{outcomes}
\ExpandOutcome{f}{1}
\ExpandOutcome{FH}{1}
\ExpandOutcome{HU}{1}
\end{outcomes}

\begin{unit}{La Oratoria}{SaberHablar2008,ManuelOratoria2008}{6}{1}
\begin{topics}
	\item La Oratoria
	\item La funci�n de la palabra.
	\item El proceso de la comunicaci�n.
	\item Bases racionales y emocionales de la oratoria.
	\item Fuentes de conocimiento.
\end{topics}
\begin{unitgoals}
	\item El alumno conocer� y se conectar� con las bases de la oratoria como fundamento te�rico y pr�ctico.
\end{unitgoals}
\end{unit}

\begin{unit}{El Orador}{SaberHablar2008,ManuelOratoria2008}{6}{1}
\begin{topics}
	\item Cualidades de un buen orador.
	\item Normas para hablar en clase.
	\item Oradores con historia y su ejemplo.
	\item El cuerpo humano como instrumento de comunicaci�n: cuerpo y voz.
\end{topics}
\begin{unitgoals}
	\item Que el alumno logre conocimientos y habilidades de la comunicaci�n oral mediante la experiencia de grandes oradores y la suya propia y pueda expresarse en p�blico en forma eficiente, inteligente y agradable.
\end{unitgoals}
\end{unit}

\begin{unit}{El Discurso}{Altamirano2008,ComoHablarBienenPublico2004}{6}{1}
\begin{topics}
	\item Composiciones - Los primeros discursos en clase.
	\item El prop�sito del discurso. El auditorio.
	\item Clases de discurso.
	\item Redacci�n de discurso. Lectura
\end{topics}
\begin{unitgoals}
	\item Que el alumno sea capaz de producir sus propios discursos de manera correcta, coherente y oportuna teniendo en cuenta su prop�sito y hacia quien los dirige. 
\end{unitgoals}
\end{unit}

\begin{unit}{Material de Apoyo}{ManuelOratoria2008}{3}{1}
\begin{topics}
	\item Las fichas, apuntes, citas.
	\item Recursos t�cnicos.
\end{topics}
\begin{unitgoals}
	\item Que el alumno conozca y utilice material de apoyo en forma adecuada y correcta para hacer mas eficiente su discurso.
\end{unitgoals}
\end{unit}

\begin{unit}{Los debates}{Altamirano2008,SaberHablar2008,ComoHablarBienenPublico2004}{15}{1}
\begin{topics}
	\item Situaciones y debates.
	\item La refutaci�n.
	\item Conciliaci�n de ideas.
	\item Principales temas de debates.
	\item Consenso.
	\item Debatir, hablar y escuchar.
	\item Intercambio de opiniones.
	\item Cr�tica y autocr�tica.
\end{topics}
\begin{unitgoals}
	\item Dotar al alumno de t�cnicas vinculadas a la sustentaci�n de ideas y manejo de debates.
\end{unitgoals}
\end{unit}

\begin{unit}{La persuaci�n}{SaberHablar2008,ComoHablarBienenPublico2004,ManuelOratoria2008}{15}{1}
\begin{topics}
	\item Persuaci�n y argumentaci�n.
	\item Discursos persuasivos.
	\item Motivar a la acci�n.
	\item Convicci�n y refutaci�n.
	\item Comunicaci�n en grupo.
	\item Proceso, din�mica y t�cnicas de la persuaci�n.
\end{topics}
\begin{unitgoals}
	\item Manejar herramientas persuasivas y t�cnicas de comunicaci�n grupal.
\end{unitgoals}
\end{unit}

\begin{coursebibliography}
\bibfile{GeneralEducation/FG203}
\end{coursebibliography}
\end{syllabus}
