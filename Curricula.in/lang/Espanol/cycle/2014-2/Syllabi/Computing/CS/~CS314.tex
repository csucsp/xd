\begin{syllabus}

\course{CS314. Algoritmos Paralelos}{Obligatorio}{CS314}

\begin{justification}
Las arquitecturas de computadores est�n tendiendo a incluir cada vez m�s n�cleos 
y/o procesadores por m�quina como m�todo de incrementar la capacidad computacional
de cada unidad. La posibilidad de realizar m�ltiples tareas simultaneamente mediante hardware 
no es inmediatamente traducida al software, pues las aplicaciones deben 
ser dise�adas para aprovechar estas nuevas capacidades, mediante el uso de hebras y/o procesos.
\end{justification}

\begin{goals}
\item Que el alumno sea capaz de crear aplicaciones paralelas de mediana complejidad aprovechando eficientemente m�quinas con m�ltiples n�cleos.
\item Que el alumno sea capaz de comparar aplicaciones secuenciales y paralelas.
\item Que el alumno sea capaz de convertir, cuando la situaci�n lo amerite, aplicaciones secuenciales a paralelas de forma eficiente.
\end{goals}

\begin{outcomes}
\ExpandOutcome{a}{3}
\ExpandOutcome{b}{4}
\ExpandOutcome{c}{1}
\ExpandOutcome{g}{1}
\ExpandOutcome{h}{1}
\ExpandOutcome{i}{1}
\ExpandOutcome{j}{1}
\end{outcomes}

\begin{unit}{\CNParallelComputationDef}{progpara}{5}{1}
      \CNParallelComputationAllTopics %% Esto esta cubriendo todos los topicos.
      \CNParallelComputationAllObjectives
\end{unit}

\begin{unit}{\ARMultiprocessingDef}{progpara}{5}{1}
      \ARMultiprocessingAllTopics
      \ARMultiprocessingAllObjectives
\end{unit}

\begin{unit}{\ALParallelAlgorithmsDef}{progpara}{3}{1}
      \ALParallelAlgorithmsAllTopics %% Esta parte en realidad esta embebida en la primera.
      \ALParallelAlgorithmsAllObjectives
\end{unit}

\begin{unit}{Modelos de Threads con PTHREADs}{pthread,progpara}{0}{1}
\begin{topics}
         \item �Qu� es una hebra?
         \item �Qu� es  pthread?
         \item Dise�ando programas con pthreads.
         \item Creacu�n y manejo de hebras.
         \item Sincronizaci�n de hebras con mutex.
\end{topics}

\begin{unitgoals}
	\item Entender los distintos modelos de programaci�n paralela.
	\item Conocer ventajas y desventajas de los distintos modelos de programaci�n paralela.
\end{unitgoals}
\end{unit}

\begin{unit}{Modelos de Threads con OpenMP}{openmp,progpara}{0}{1}
\begin{topics}
         \item �Qu� es OpenMP?
         \item El modelo de programaci�n OpenMP.
         \item Directivas de OpenMP.
         \item Constructores de trabajo compartido.
         \item Constructores de Tareas.
         \item Constructores de sincronizaci�n.
	 \item Manejo de datos privados y compartidos.
\end{topics}

\begin{unitgoals}
	\item Implementar programas multihebras por medio de OpenMP.
	\item Entender y aplicar conceptos de sincronizaci�n y trabajo compartido.
\end{unitgoals}
\end{unit}

\begin{unit}{Modelo de programaci�n mediante paso de Mensajes con MPI}{mpi,progpara}{0}{1}
\begin{topics}
         \item �Qu� es MPI?
         \item Rutinas de administraci�n de ambiente.
         \item Rutinas de comunicaci�n punto a punto.
         \item Rutinas de comunicaci�n colectiva.
         \item Tipos de datos derivados.
         \item Rutinas de administraci�n del comunicador y de grupo.
	 \item Topolog�a virtual.
\end{topics}

\begin{unitgoals}
	\item Implementar programas multihebras por medio de OpenMP.
	\item Entender y aplicar conceptos de sincronizaci�n y trabajo compartido.
\end{unitgoals}
\end{unit}

\begin{unit}{{\it Threading Building Blocks (TBB)}}{tbb,progpara}{0}{1}
\begin{topics}
      \item Bucles Simples Paralelos.
      \item Bucles Complejos Paralelos.
      \item Cancelaci�n y Exepciones.
      \item Contenedores paralelos. 
\end{topics}

\begin{unitgoals}
	\item Entender y aplicar el modelo de datos paralelos utilizando la herramienta TBB.
\end{unitgoals}
\end{unit}

\begin{coursebibliography}
\bibfile{Computing/CS/CS314}
\end{coursebibliography}

\end{syllabus}
