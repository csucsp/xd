\begin{syllabus}

\course{CS112. Ciencia de la Computaci�n I}{Obligatorio}{CS112}

\begin{justification}
Este es el segundo curso en la l�nea de programaci�n de los cursos introductorios a la Ciencia de la Computaci�n.
El curso servir� como puente entre el paradigma imperativo y el orientado a objetos, adem�s introducir� a
los participantes en los diversos temas del �rea de computaci�n como: algoritmos, estructuras de datos, ingenier�a del
software, etc.
\end{justification}

\begin{goals}
\item Introducir al alumno a los fundamentos del paradigma orientado a objetos, permitiendo asimilar los conceptos
necesarios para desarrollar sistemas de informaci�n.
\end{goals}

\begin{outcomes}
    \item \ShowOutcome{a}{3}
    \item \ShowOutcome{c}{3}
    \item \ShowOutcome{h}{1}
    \item \ShowOutcome{i}{2}
\end{outcomes}

\begin{competences}
    \item \ShowCompetence{C1}{a} 
    \item \ShowCompetence{C2}{h} 
    \item \ShowCompetence{C3}{c} 
    \item \ShowCompetence{C23}{i}
    \item \ShowCompetence{C24}{i}
    \item \ShowCompetence{CS1}{c}
    \item \ShowCompetence{CS2}{c}
    \item \ShowCompetence{CS6}{a}
\end{competences}

\begin{unit}{\SPHistory}{Visi�n General de los Lenguajes de Programaci�n}{Stroustrup2013,Deitel2017}{1}{C1}
    \begin{topics}%
        \item Breve revisi�n de los paradigmas de programaci�n.
        \item Comparaci�n entre programaci�n funcional y programaci�n imperativa.
        \item Historia de los lenguajes de programaci�n.
    \end{topics}
    \begin{learningoutcomes}%Familiarity,Usage,Assessment
        \item \SPHistoryLODiscussTheForLanguage [\Familiarity]
    \end{learningoutcomes}
\end{unit}

\begin{unit}{\OSVirtualMachines}{}{Stroustrup2013,Deitel2017}{1}{C2, CS6}
    \begin{topics}%
        \item El concepto de m�quina virtual.
        \item \OSVirtualMachinesTopicTypes.
        \item Lenguajes intermedios.
    \end{topics}
    \begin{learningoutcomes}%Familiarity,Usage,Assessment
        \item \OSVirtualMachinesLOExplainTheVirtual [\Familiarity]
        \item \OSVirtualMachinesLODifferentiateEmulation [\Familiarity]
        \item \OSVirtualMachinesLOEvaluateVirtualization [\Assessment]
    \end{learningoutcomes}
\end{unit}

\begin{unit}{\PLBasicTypeSystems}{}{Stroustrup2013,Deitel2017}{2}{C1,C2,CS1}
    \begin{topics}%
        \item \PLBasicTypeSystemsTopicA        
        \item Declaraci�n de modelos (enlace, visibilidad, alcance y tiempo de vida).
        \item Vista general del chequeo de tipos.
    \end{topics}
    \begin{learningoutcomes}
        \item \PLBasicTypeSystemsLOForBoth [\Familiarity] 
        \item \PLBasicTypeSystemsLOForA [\Familiarity] 
        \item \PLBasicTypeSystemsLODescribeExamples [\Familiarity] 
        \item \PLBasicTypeSystemsLOForMultiple [\Usage] 
        \item \PLBasicTypeSystemsLOGiveAnThat [\Familiarity] 
        \item \PLBasicTypeSystemsLOUseTypes [\Usage] 
        \item \PLBasicTypeSystemsLOExplainHowDefine [\Familiarity] 
        \item \PLBasicTypeSystemsLOWriteDown [\Usage] 
        \item \PLBasicTypeSystemsLOExplainWhyType [\Familiarity] 
        \item \PLBasicTypeSystemsLODefineAndPieces [\Usage] 
        \item \PLBasicTypeSystemsLODiscussTheGenerics [\Familiarity] 
        \item \PLBasicTypeSystemsLOExplainMultiple [\Familiarity] 
    \end{learningoutcomes}
\end{unit}

\begin{unit}{\SDFFundamentalProgrammingConcepts}{}{Stroustrup2013,Deitel2017}{6}{C1,C2,CS2}
    \begin{topics}%
        \item \SDFFundamentalProgrammingConceptsTopicBasic
        \item \SDFFundamentalProgrammingConceptsTopicVariables
        \item \SDFFundamentalProgrammingConceptsTopicExpressions
        %\item \SDFFundamentalProgrammingConceptsTopicSimple
        \item \SDFFundamentalProgrammingConceptsTopicConditional
        %\item \SDFFundamentalProgrammingConceptsTopicFunctions
    \end{topics}
    \begin{learningoutcomes}
        \item \SDFFundamentalProgrammingConceptsLOAnalyzeAndBehavior [\Assessment]
        \item \SDFFundamentalProgrammingConceptsLOIdentifyAndOf [\Familiarity]
        \item \SDFFundamentalProgrammingConceptsLOWritePrograms [\Usage]
        \item \SDFFundamentalProgrammingConceptsLOModify [\Usage]
        \item \SDFFundamentalProgrammingConceptsLODesignImplement [\Usage]
        %\item \SDFFundamentalProgrammingConceptsLOWriteAUses [\Usage]
        \item \SDFFundamentalProgrammingConceptsLOChooseAppropriateIteration [\Assessment]
        %\item \SDFFundamentalProgrammingConceptsLODescribeTheRecursion [\Familiarity]
        %\item \SDFFundamentalProgrammingConceptsLOIdentifyTheAndCase [\Assessment]
    \end{learningoutcomes}
\end{unit}

\begin{unit}{Funciones}{}{Stroustrup2013,Deitel2017}{10}{C1,C2,CS2}
    \begin{topics}%
        \item \SDFFundamentalProgrammingConceptsTopicFunctions
        \item Paso de par�metros
        \item Sobrecarga en funciones
        \item Fundamentos de la recursidad 
        \item Conceptos de plantillas en funciones
      

    \end{topics}
    \begin{learningoutcomes}
        \item  \SDFFundamentalProgrammingConceptsLODesignImplement [\Usage]
        \item  Entiende y aplica el concepto de paso de par�metros a una funci�n, tanto por valor como por referencia.[\Usage]
        \item  Identifica y aplica el concepto de sobrecarga de funciones.[\Usage]
        \item  \SDFFundamentalProgrammingConceptsLODescribeTheRecursion [\Familiarity]
        \item  Dise�a, implementa y aplica el concpeto de plantillas asociado a la ncesidad de crear funciones gen�ricas.[\Usage]

    \end{learningoutcomes}
\end{unit}

\begin{unit}{Arreglos y Punteros}{}{Stroustrup2013,Deitel2017}{10}{C1,C2,CS2}
    \begin{topics}%
        \item Definici�n de arreglos
        \item Arreglos multidimensionales
        \item Fundamentos sobre punteros
        \item Administraci�n din�mica de memoria 
        \item Conceptos avanzados de Punteros 
      

    \end{topics}
    \begin{learningoutcomes}
        \item  Entiende e implementa arreglos unidimensionales. [\Familiarity]
        \item  Dise�a y aplica el concepto de arreglos multidimensionales.[\Usage]
        \item  Entiende y aplica el concepto de referencias y punteros.[\Familiarity]
        \item  Entiende, aplica y evalua la relaci�n entre punteros y arreglos.[\Assessment]
        \item  Entiende e implementa la gesti�n din�mica de la memoria. Diferenciando las regiones de memoria: heap y stack. [\Assessment]
        \item  Dise�a, implementa y evalua el concepto de puntero a puntero, puntero a funci�n, entre otros conceptos.[\Assessment]

    \end{learningoutcomes}
\end{unit}

\begin{unit}{\PLObjectOrientedProgramming}{}{Stroustrup2013,Deitel2017}{10}{C2,C24,CS1,CS2}
    \begin{topics}%
        \item \PLObjectOrientedProgrammingTopicObject
        \item \PLObjectOrientedProgrammingTopicObjectOriented
        \item \PLObjectOrientedProgrammingTopicDefinition
        \item \PLObjectOrientedProgrammingTopicSubclasses
        \item \PLObjectOrientedProgrammingTopicSubtyping
        \item \PLObjectOrientedProgrammingTopicUsing
        \item \PLObjectOrientedProgrammingTopicDynamic

    \end{topics}
    \begin{learningoutcomes}
        \item \PLObjectOrientedProgrammingLODesignAndClass [\Usage]
        \item \PLObjectOrientedProgrammingLOUseSubclassing [\Usage]
        \item \PLObjectOrientedProgrammingLOCorrectly [\Usage]
        \item \PLObjectOrientedProgrammingLOCompareAndThe [\Assessment]
        \item \PLObjectOrientedProgrammingLOExplainTheObject [\Familiarity]
        \item \PLObjectOrientedProgrammingLOUseObject [\Usage]
        \item \PLObjectOrientedProgrammingLODefineAndAnd [\Usage]

    \end{learningoutcomes}
\end{unit}

\begin{unit}{Plantillas y STL}{}{Stroustrup2013,Deitel2017}{10}{C1,C2,CS2}
    \begin{topics}%
        \item Definici�n de plantillas en clases
        \item Conceptos b�sicos sobre la Standard Template Library (STL)
      
    \end{topics}
    \begin{learningoutcomes}
        \item  Entiende los conceptos de plantillas en clases. [\Familiarity]
        \item  Implementa y crea nuevos tipos de datos gen�ricos. [\Usage]
        \item  Entiende las estructuras b�sicas de la STL. [\Familiarity]
        \item  Usa las estructuras de datos b�sicas como: pila, cola, lista, vector contenidos en la STL. [\Usage]

    \end{learningoutcomes}
\end{unit}

\begin{unit}{Conceptos Avanzados}{}{Stroustrup2013,Deitel2017}{10}{C1,C2,CS2}
    \begin{topics}%
        \item Definici�n de sobrecarga de operadores
        \item Manipulaci�n de entrada y salida de datos (I/O)
        \item Patrones de dise�o
      
    \end{topics}
    \begin{learningoutcomes}
        \item  Entiende los conceptos de sobrecarga de operadores. [\Familiarity]
        \item  Implementa la sobrecarga de operadores permitidos en el lenguaje de programaci�n. [\Usage]
        \item  Entiende los conceptos de manipulaci�n de archivos. [\Familiarity]
        \item  Crea programas de lectura y escrita en archivos. [\Usage]
        \item  Entiende los conceptos de patrones de dise�o. [\Familiarity]

    \end{learningoutcomes}
\end{unit}



\begin{coursebibliography}
\bibfile{Computing/CS/CS112}
\end{coursebibliography}

\end{syllabus}

%\end{document}
