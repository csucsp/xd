\begin{syllabus}

\course{CS311. Programaci�n Competitiva}{Obligatorio}{CS311}

\begin{justification}
La Programaci�n Competitiva combina retos de solucionar problemas con la diversi�n de competir con otras personas. Ense�a a los participantes a pensar m�s r�pido y desarrollar habilidades para resolver problemas, que son de gran demanda en la industria. 
Este curso ense�ar� la resoluci�n de problemas algor�tmicos de manera r�pida combinando la teor�a de algoritmos y estructuras de datos con la pr�ctica la soluci�n de los problemas.
\end{justification}

\begin{goals}
\item Que el alumno utilice t�cnicas de estructuras de datos y algoritmos complejos.
\item Que el alumno aplique los conceptos aprendidos para la aplicaci�n sobre un problema real.
\item Que el alumno investigue la posibilidad de crear un nuevo algoritmo y/o t�cnica nueva para resolver un problema real.
\end{goals}

\begin{outcomes}
    \item \ShowOutcome{a}{2}
    \item \ShowOutcome{b}{2}
    \item \ShowOutcome{i}{2}
    \item \ShowOutcome{j}{2}
\end{outcomes}

\begin{competences}
    \item \ShowCompetence{C1}{a,b}
    \item \ShowCompetence{C24}{i,j}
\end{competences}

\begin{unit}{}{Primera Unidad}{Cormen2009}{20}{C24,C1}
\begin{topics}
        \item Estructura de datos
        \item Programaci�n din�mica
        \item Algoritmos basados en grafos
        \item Geometr�a computacional
        \item Algoritmos de ordenamiento
\end{topics}
\begin{learningoutcomes}
        \item Aprender a seleccionar los algoritmos adecuados para un problema dado, integrando m�ltiples algoritmos para la soluci�n de un problema complejo. [\Usage]
	\item Dise�ar nuevos algoritmos para la resoluci�n de problemas del mundo real.[\Usage]
\end{learningoutcomes}
\end{unit}



\begin{coursebibliography}
\bibfile{Computing/CS/CS311}
\end{coursebibliography}

\end{syllabus}
