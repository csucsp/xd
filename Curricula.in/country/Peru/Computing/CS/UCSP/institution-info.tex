\newcommand{\DocumentVersion}{2016}
\newcommand{\fecha}{\today}
\newcommand{\city}{Arequipa\xspace}
\newcommand{\country}{Per�\xspace}
\newcommand{\dictionary}{Espa�ol\xspace}
\newcommand{\GraphVersion}{2\xspace}

%newcommand{\CurriculaVersion}{1\xspace} % Malla 2006: 1, Malla 2016: 2
%newcommand{\YYYY}{2006\xspace}          % Plan 2006
%newcommand{\Range}{7-10}                % Plan 2016 1-8, Plan 2006 7-10

\newcommand{\CurriculaVersion}{2016\xspace} % Malla 2006: 1, Malla 2016: 2 i.e ../Curricula.in/lang/Espanol/CS.tex/CS2016-dependencies.tex
\newcommand{\YYYY}{2016\xspace}          % Plan 2006, 2010, 2016
\newcommand{\Range}{1-9}                % Plan 2016 1-8, Plan 2006 7-10

\newcommand{\Semester}{2020-I\xspace}
\newcommand{\equivalences}{2006,2010} %  {2006,2010}

% convert ./fig/UCSP.jpg ./html/img3.png
% cp ./fig/big-graph-curricula.png ./html/img18.png
% convert ../Curricula2.0.out/Peru/CS-UCSP/cycle/2014-1/Plan2016/fig/UCSP.jpg ../Curricula2.0.out/Peru/CS-UCSP/cycle/2014-1/Plan2016/html/img3.png
% cp ../Curricula2.0.out/Peru/CS-UCSP/cycle/2014-1/Plan2016/fig/big-graph-curricula.png ../Curricula2.0.out/Peru/CS-UCSP/cycle/2014-1/Plan2016/html/img18.png

\newcommand{\OutcomesList}{a,b,c,d,e,f,g,h,i,j,k,l,m,n,�,o}
\newcommand{\logowidth}{20cm}

\newcommand{\University}{Universidad Cat�lica San Pablo\xspace}
\newcommand{\InstitutionURL}{\htmladdnormallink{http://www.ucsp.edu.pe}{http://www.ucsp.edu.pe}\xspace}
\newcommand{\underlogotext}{}
\newcommand{\FacultadName}{}
\newcommand{\DepartmentName}{Ciencia de la Computaci�n\xspace}
\newcommand{\SchoolFullName}{Escuela Profesional de Ciencia de la Computaci�n\xspace}
\newcommand{\SchoolFullNameBreak}{Escuela Profesional de \\Ciencia de la Computaci�n\xspace}
\newcommand{\SchoolShortName}{Ciencia de la Computaci�n\xspace}
\newcommand{\SchoolAcro}{EPCC\xspace}
\newcommand{\SchoolURL}{\href{http://cs.ucsp.edu.pe}{http://cs.ucsp.edu.pe}\xspace}

\newcommand{\GradoAcademico}{Bachiller en Ciencia de la Computaci�n\xspace}
\newcommand{\TituloProfesional}{Licenciado en Ciencia de la Computaci�n\xspace}
\newcommand{\GradosyTitulos}%
{\begin{description}%
\item [Grado Acad�mico: ] \GradoAcademico\xspace y%
\item [Titulo Profesional: ] \TituloProfesional%
\end{description}


}

\newcommand{\doctitle}{Plan Curricular \YYYY\xspace del \SchoolFullName\\ \SchoolURL}

\newcommand{\AbstractIntro}{Este documento representa el informe final de la nueva 
malla curricular \YYYY de la \SchoolFullName de la \University (\textit{\InstitutionURL}) 
en la ciudad de \city-\country.}

\newcommand{\OtherKeyStones}%
{Un pilar que merece especial consideraci�n en el caso de la \University es el aspecto de 
valores humanos, b�sicos y cristianos debido a que forman parte fundamental 
de los lineamientos b�sicos de la existencia de la instituci�n.\xspace}

\newcommand{\profile}{%
El perfil profesional puede ser mejor entendido a partir de
\OnlyMainDoc{la Fig. \ref{fig.cs} (P�g. \pageref{fig.cs})}\OnlyPoster{las figuras del lado derecho}. 
Este profesional tiene como objetivo principal ser el impulsor del desarrollo de nuevas 
tecnolog�as computacionales con calidad internacional que puedan ser �tiles a nivel local, nacional e internacional.
Nuestro perfil profesional tambi�n est� orientado a ser generador de puestos de empleo a trav�s de la innovaci�n permanente. 
Nuestra formaci�n profesional tiene 3 pilares fundamentales: 
un contenido computacional de acuerdo a normas internacionales (CS2013), una orientaci�n marcada a la innovaci�n ambos enriquecidos por una s�lida
Formaci�n Humana.
}

\newcommand{\mission}{La Universidad Cat�lica San Pablo es una comunidad acad�mica animada por las orientaciones y vida de la Iglesia Cat�lica que, 
a la luz de la fe y con el esfuerzo de la raz�n, busca la verdad y promueve la formaci�n integral de la persona mediante actividades 
como la investigaci�n, la ense�anza y la extensi�n, para contribuir con la 
configuraci�n de la cultura conforme a la identidad y despliegue propios del ser humano.\xspace}

\newcommand{\HTMLFootnote}{{Generado por <A HREF='http://socios.spc.org.pe/ecuadros/'>Ernesto Cuadros-Vargas</A>, <A HREF='http://www.ucsp.edu.pe/'>Universidad Cat�lica San Pablo, Arequipa-Per�</A><BR> basado en el modelo de la <I>Computing Curricula CS2013</I> de <A HREF='http://www.computer.org/'>IEEE-CS</A>/<A HREF='http://www.acm.org/'>ACM</A> y el la propuesta de la <A HREF='http://www.spc.org.pe/'>Sociedad Peruana de Computaci�n</A>,  }}

\newcommand{\Copyrights}{Generado por Ernesto Cuadros-Vargas, Universidad Cat�lica San Pablo, Arequipa-Per� basado en la propuesta curricular de la Sociedad Peruana de Computaci�n (http://www.spc.org.pe/) y \\en la {\it Computing Curricula} CS2013 de IEEE-CS (http://www.computer.org) y ACM (http://www.acm.org/)}
