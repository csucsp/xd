\begin{syllabus}

\course{CS107. �lgebra Abstracta}{Obligatorio}{CS107}

\begin{justification}
El �lgebra abstracta tiene un lado pr�ctico que explotaremos para
comprender en profundidad temas de computaci�n como criptograf�a y
�lgebra relacional.
\end{justification}

\begin{goals}
\item Conocer las t�cnicas y m�todos de encriptaci�n de datos.
\end{goals}

\begin{outcomes}
\ExpandOutcome{a}{1}
\ExpandOutcome{b}{1}
\ExpandOutcome{j}{1}
\end{outcomes}

\begin{unit}{\ALCryptographicAlgorithmsDef}{Grimaldi97, Scheinerman01}{20}{1}
    \ALCryptographicAlgorithmsAllTopics
    \ALCryptographicAlgorithmsAllObjectives
\end{unit}

\begin{unit}{\IMRelationalDatabasesDef}{Grassmann97}{20}{1}
   \begin{topics}
         %\item \IMRelationalDatabasesTopicMapeo
         \item \IMRelationalDatabasesTopicEntity
         \item \IMRelationalDatabasesTopicRelational
   \end{topics}

   \begin{unitgoals}
         %\item \IMRelationalDatabasesObjONE
         %\item \IMRelationalDatabasesObjTWO
         \item \IMRelationalDatabasesObjTHREE
         \item \IMRelationalDatabasesObjFOUR
         \item \IMRelationalDatabasesObjFIVE
   \end{unitgoals}
\end{unit}

\begin{unit}{Teor�a de N�meros}{Grimaldi97, Scheinerman01}{20}{1}
   \begin{topics}
      \item Teor�a de los n�meros
      \item Aritm�tica  Modular
      \item Teorema del Residuo Chino
      \item Factorizaci�n
      \item Grupos, teor�a de la codificaci�n y m�todo de enumeraci�n de Polya
      \item Cuerpos finitos y dise�os combinatorios
   \end{topics}

   \begin{unitgoals}
      \item Establecer la importancia de la teor�a de n�meros en la criptograf�a
      \item Utilizar las propiedades de las estructuras algebraicas en el estudio de la teor�a algebraica de c�digos
   \end{unitgoals}
\end{unit}

\begin{coursebibliography}
\bibfile{Computing/CS/CS105}
\end{coursebibliography}

\end{syllabus}

%\end{document}
