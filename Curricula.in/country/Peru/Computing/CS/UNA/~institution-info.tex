\newcommand{\DocumentVersion}{V2.0}
\newcommand{\fecha}{6 de Mayo de 2010}
\newcommand{\YYYY}{2011\xspace}
\newcommand{\Semester}{2011-1\xspace}
\newcommand{\city}{Abancay\xspace}
\newcommand{\country}{Per�\xspace}
\newcommand{\dictionary}{Espa�ol\xspace}
\newcommand{\GraphVersion}{2\xspace}
\newcommand{\CurriculaVersion}{2\xspace}
\newcommand{\OutcomesList}{a,b,c,d,e,f,g,h,i,j,k,l,m,HU,FH,TASDSH}
\newcommand{\logowidth}{7cm}

\newcommand{\University}{Universidad Nacional del Altiplano\xspace}
\newcommand{\InstitutionURL}{http://www.una.edu.pe\xspace}
\newcommand{\underlogotext}{}
\newcommand{\FacultadName}{Computaci�n\xspace}
\newcommand{\DepartmentName}{Ciencia de la Computaci�n\xspace}
\newcommand{\SchoolFullName}{Escuela Acad�mico Profesional Ciencia de la Computaci�n}
\newcommand{\SchoolFullNameBreak}{Facultad de Computaci�n\\ Escuela Acad�mico  Profesional de\\ Ciencia de la Computaci�n\xspace}
\newcommand{\SchoolShortName}{Ciencia de la Computaci�n\xspace}
\newcommand{\SchoolAcro}{EPCC\xspace}
\newcommand{\SchoolURL}{http://cs.una.edu.pe}

\newcommand{\GradoAcademico}{Bachiller en Ciencia de la Computaci�n\xspace}
\newcommand{\TituloProfesional}{Ciencia de la Computaci�n\xspace}
\newcommand{\GradosyTitulos}%
{\begin{description}%
\item [Grado Acad�mico: ] \GradoAcademico\xspace y%
\item [Titulo Profesional: ] \TituloProfesional%
\end{description}%
}

\newcommand{\doctitle}{Plan Curricular \YYYY\xspace del \SchoolFullName\\ \SchoolURL}

\newcommand{\AbstractIntro}{Este documento representa el informe final de la nueva malla curricular \YYYY del \SchoolFullName de
la \University (\textit{\InstitutionURL}) en la ciudad de \city-\country.}

\newcommand{\OtherKeyStones}{}
% {Un pilar que merece especial consideraci�n en el caso de la \University es el aspecto de 
% valores humanos, b�sicos y cristianos debido a que forman parte fundamental 
% de los lineamientos b�sicos de la existencia de la instituci�n.\xspace}

\newcommand{\profile}{%
El perfil profesional de este programa profesional puede ser mejor entendido a partir de
\OnlyMainDoc{la Fig. \ref{fig.cs} (P�g. \pageref{fig.cs})}\OnlyPoster{las figuras del lado derecho}. 
Este profesional tiene como centro de su estudio a la computaci�n. Es decir, tiene a la computaci�n 
como fin y no como medio. De acuerdo a la definici�n de esta �rea, este profesional est� llamado 
directamente a ser un impulsor del desarrollo de nuevas t�cnicas computacionales que 
puedan ser �tiles a nivel local, nacional e internacional.

Nuestro perfil profesional est� orientado a ser generador de puestos de empleo a trav�s de la innovaci�n permanente. 
Nuestra formaci�n profesional tiene 3 pilares fundamentales: 
Formaci�n Humana, un contenido de acuerdo a normas internacionales y una orientaci�n marcada a la innovaci�n.
}

\newcommand{\HTMLFootnote}{Generado por <A HREF='http://socios.spc.org.pe/ecuadros/'>Ernesto Cuadros-Vargas</A> <ecuadros AT spc.org.pe>, basado en el modelo de la <A HREF='http://www.spc.org.pe/'>Sociedad Peruana de Computaci�n, Per�</A><BR> y en la Computing Curricula de <A HREF='http://www.computer.org/'>IEEE-CS</A>/<A HREF='http://www.acm.org/'>ACM</A>}

\newcommand{\Copyrights}{Generado por Ernesto Cuadros-Vargas (ecuadros AT spc.org.pe), Sociedad Peruana de Computaci�n (http://www.spc.org.pe/), Universidad Cat�lica San Pablo (http://www.ucsp.edu.pe) \country basado en la {\it Computing Curricula} de IEEE-CS (http://www.computer.org) y ACM (http://www.acm.org/)}
