% Responsable : Luis D�az Basurco
% Sumilla de  : Estad�stica y Probabilidades
% Versi�n     : 1

\begin{syllabus}

\course{MA308. An�lisis Exploratorio de Datos Espaciales}{Obligatorio}{MA308}

\begin{justification}
Provee de una introducci�n a la teor�a de las probabilidades e inferencia estad�stica con aplicaciones, necesarias en el an�lisis de datos, dise�o de modelos aleatorios y toma de decisiones.
\end{justification}

\begin{goals}
\item Que el alumno aprenda a utilizar las herramientas de la estad�stica para tomar decisiones ante situaciones de incertidumbre.
\item Que el alumno aprenda a obtener conclusiones a partir de datos experimentales.
\item Que el alumno pueda extraer conslusiones �tiles sobre la totalidad de una poblaci�n bas�ndose en informaci�n. recolectada
\end{goals}

\begin{outcomes}
   \item \ShowOutcome{a}{2}
   \item \ShowOutcome{i}{2}
   \item \ShowOutcome{j}{3}
\end{outcomes}

\begin{competences}
    \item \ShowCompetence{C1}{a} 
    \item \ShowCompetence{CS6}{i}
    \item \ShowCompetence{CS2}{j}
\end{competences}


\begin{unit}{}{Estad�stica descriptiva}{Mendenhall97}{6}{C1}
\begin{topics}
      \item Presentaci�n de datos
      \item Medidas de localizaci�n central
      \item Medidas de dispersi�n
   \end{topics}

   \begin{learningoutcomes}
      \item Presentar resumir y describir datos. [\Usage]
   \end{learningoutcomes}
\end{unit}

\begin{unit}{}{Probabilidades}{Meyer70}{6}{C1}
\begin{topics}
      \item Espacios muestrales y eventos
      \item Axiomas y propiedades de probabilidad
      \item Probabilidad condicional
      \item Independencia,
      \item Teorema de Bayes
   \end{topics}
   \begin{learningoutcomes}
      \item Identificar espacios aleatorios [\Usage]
      \item dise�ar  modelos probabil�sticos [\Usage]
      \item Identificar eventos como resultado de un  [\Usage]experimento aleatorio [\Usage]
      \item Calcular la probabilidad de ocurrencia de un evento [\Usage]
      \item Hallar la probabilidad usando condicionalidad, independencia y Bayes [\Usage]
   \end{learningoutcomes}
\end{unit}

\begin{unit}{}{Variable aleatoria}{Meyer70,Devore98}{6}{CS6}
\begin{topics}
      \item Definici�n y tipos de variables aleatorias 
      \item Distribuci�n de probabilidades
      \item Funciones densidad
      \item Valor esperado
      \item Momentos
   \end{topics}

   \begin{learningoutcomes}
      \item Identificar variables aleatorias que describan un espacio muestra [\Usage]
      \item Construir la distribuci�n o funci�n de densidad. [\Usage]
      \item Caracterizar distribuciones o funciones densidad conjunta. [\Usage]
   \end{learningoutcomes}
\end{unit}

\begin{unit}{}{Distribuci�n de probabilidad discreta y continua}{Meyer70,Devore98}{6}{CS6}
\begin{topics}
      \item Distribuciones de probabilidad b�sicas
      \item Densidades de probabilidad b�sicas
      \item Funciones de variable aleatoria
   \end{topics}

   \begin{learningoutcomes}
      \item Calcular probabilidad de una variable aleatoria con distribuci�n o funci�n densidad [\Usage]
      \item Identificar la distribuci�n o funci�n densidad que describe un problema aleatorio [\Usage]
      \item Probar propiedades de distribuciones o funciones de densidad [\Usage]
   \end{learningoutcomes}
\end{unit}

\begin{unit}{}{Distribuci�n de probabilidad conjunta}{Meyer70,Devore98}{6}{CS2}
\begin{topics}
      \item Variables aleatorias distribuidas conjuntamente 
      \item Valores esperados, covarianza y correlaci�n
      \item Las estad�sticas y sus distribuciones
      \item Distribuci�n de medias de muestras
      \item Distribuci�n de una combinaci�n lineal

   \end{topics}
   \begin{learningoutcomes}
      \item Encontrar la distribuci�n conjunta de dos variables aleatorias discretas o continuas [\Usage]
      \item Hallar las distribuciones marginales o condicionales de variables aleatorias conjuntas [\Usage]
      \item Determinar dependencia o independencia de variables aleatorias [\Usage]
      \item Probar propiedades que son consecuencia del teorema  del l�mite central [\Usage]
   \end{learningoutcomes}
\end{unit}

\begin{unit}{}{Inferencia estad�stica}{Meyer70,Devore98}{6}{CS2}
\begin{topics}
      \item Estimaci�n estad�stica
      \item Prueba de hip�tesis
      \item Prueba de hip�tesis usando ANOVA
   \end{topics}

   \begin{learningoutcomes}
      \item Probar si un estimador es insesgado, consistente o suficiente [\Usage]
      \item Hallar intervalo intervalos de confianza para estimar par�metros [\Usage]
      \item Tomar decisiones de par�metros en base a pruebas de hip�tesis [\Usage]
      \item Probar hip�tesis usando ANOVA [\Usage]
   \end{learningoutcomes}
\end{unit}







\begin{coursebibliography}
\bibfile{BasicSciences/MA308}
\end{coursebibliography}

\end{syllabus}
