% Responsable : Luis D�az Basurco
% Sumilla de  : An�lisis Matem�tico III
% Versi�n     : 1

\begin{syllabus}

\course{CB201. An�lisis Matem�tico III}{Obligatorio}{CB201}

\begin{justification}
Es una extensi�n de los cursos de An�lisis Matem�tico I y An�lisis Matem�tico II, tomando en cuenta dos o m�s variables, indispensables para aquellas materias que requieren trabajar con geometr�a en curvas y superficies, as� como en procesos de b�squeda de puntos extremos.
\end{justification}

\begin{goals}
\item Diferenciar e integrar funciones vectoriales de variable real, entender y manejar el concepto de parametrizaci�n. Describir una curva en forma param�trica.
\item Describir, analizar, dise�ar y formular modelos continuos que dependen de m�s de una variable.
\item Establecer relaciones entre diferenciaci�n e integraci�n y aplicar el c�lculo diferencial e integral ala resoluci�n de problemas geom�tricos y de optimizaci�n.
\end{goals}

\begin{outcomes}
\ExpandOutcome{a}{3}
\ExpandOutcome{i}{3}
\ExpandOutcome{j}{4}
\end{outcomes}

\begin{unit}{Geometr�a en el espacio}{Apostol73,Simmons95}{8}{3}
   \begin{topics}
      \item $R^3$ como espacio eucl�deo y �lgebra.
      \item Superficies b�sicas en el espacio.
   \end{topics}

   \begin{unitgoals}
      \item Manejar el �lgebra vectorial en $R^3$
      \item Identificar tipos de superficies en el espacio
      \item Graficar superficies b�sicas
      \end{unitgoals}
\end{unit}

\begin{unit}{Curvas y parametrizaciones}{Apostol73,Simmons95}{20}{3}
   \begin{topics}
      \item Funciones vectoriales de variable real. Reparametrizaciones
      \item Diferenciaci�n e integraci�n
      \item Velocidad, aceleraci�n, curvatura, torsi�n
      \end{topics}

   \begin{unitgoals}
      \item Describir las diferentes caracter�sticas de una curva
      \end{unitgoals}
\end{unit}

\begin{unit}{Campos escalares}{Apostol73,Bartle76,Simmons95}{20}{3}
   \begin{topics}
      \item Curvas de nivel
      \item L�mites y continuidad
      \item Diferenciaci�n
      \end{topics}

   \begin{unitgoals}
      \item Graficar campos escalares
      \item Discutir la existencia de un l�mite y la continuidad de un campo escalar
      \item Calcular derivadas parciales y totales.
      \end{unitgoals}
\end{unit}

\begin{unit}{Aplicaciones}{Apostol73,Simmons95,Bartle76}{12}{3}
   \begin{topics}
      \item M�ximos y m�nimos
      \item Multiplicadores de Lagrange
      \end{topics}

   \begin{unitgoals}
      \item Interpretar la noci�n de gradiente en curvas de nivel y en superficies de nivel
      \item Usar t�cnicas para hallar extremos
      \end{unitgoals}
\end{unit}

\begin{unit}{Integraci�n M�ltiple}{Apostol73}{12}{4}
   \begin{topics}
      \item Integraci�n de Riemann
      \item Integraci�n sobre regiones
      \item Cambio de coordenadas
      \item Aplicaciones
      \end{topics}

   \begin{unitgoals}
      \item Reconocer regiones de integraci�n adecuadas
      \item Realizar cambios de coordenadas adecuados
      \item Aplicar la integraci�n m�ltiple a problemas
      \end{unitgoals}
\end{unit}

\begin{unit}{Campos vectoriales}{Apostol73}{18}{3}
   \begin{topics}
      \item Integrales de linea
      \item campos conservativos
      \item Integrales de superficie
   \end{topics}

   \begin{unitgoals}
      \item Calcular la integral de linea de campos vectoriales
      \item Reconocer campos conservativos
      \item Hallar funciones potenciales de campos conservativos
      \item Hallar integrales de superficies y aplicarlas
      \end{unitgoals}
\end{unit}



\begin{coursebibliography}
\bibfile{BasicSciences/CB201}
\end{coursebibliography}

\end{syllabus}
