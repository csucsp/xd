La computaci�n ha sufrido un desarrollo impresionante en los �ltimos 60 a�os, convirti�ndose en el motor del desarrollo cient�fico, tecnol�gico, industrial, social, econ�mico y cultural, transformando de manera significativa nuestro diario accionar.

El surgimiento del computador ha marcado una nueva era en la historia de la humanidad que era imposible de imaginar varias d�cadas atr�s. La gran cantidad de aplicaciones que se han desarrollado en los �ltimos a�os est�n transformando el desarrollo de todas las disciplinas del saber, la comercializaci�n en el �mbito globalizado en que vivimos, la manera en que nos comunicamos, los procesos de ense�anza-aprendizaje y hasta en la manera como nos entretenemos.

Para darnos una idea de la relevancia e importancia, que en nuestro pa�s ha alcanzado esta disciplina, basta mencionar que actualmente se ofrecen aproximadamente 70 carreras de Computaci�n, Inform�tica, Sistemas, a nivel nacional, sin considerar los programas de nivel T�cnico Superior No Universitario que se ofertan.

Todas estas carreras existentes tienen como centro de su estudio a la computaci�n pero lo hacen con una gran diversidad de nombres como: Ingenier�a de Sistemas, Ingenier�a de Computaci�n, Ingenier�a de Computaci�n y Sistemas, entre otros. A pesar de que todas ellas apuntan al mismo mercado de trabajo resulta por lo menos curioso que no sea posible encontrar por lo menos dos que compartan la misma curricula.

Muchos pa\'ises consideran a la computaci\'on como estrat\'egica para su desarrollo. En Per\'u, el \ac{CONCYTEC} ha recomendado al gobierno que considere a la Computaci\'on como una de las \'areas prioritarias de vinculaci\'on entre la academia e industria para fomentar la competitividad y la innovaci\'on.

Com\'unmente, durante la d\'ecada de los setenta, la Computaci\'on se desarroll\'o dentro de las Facultades de Ciencias en la mayor\'ia de las universidades estadounidenses, brit\'anicas y de otros pa\'ises. Durante la d\'ecada de los ochenta, los grupos de computaci\'on en las universidades se esforzaron por lograr una legitimidad acad\'emica en su \'ambito local. Frecuentemente, se transformaron en departamentos de Matem\'aticas y Computaci\'on, hasta finalmente dividirse en dos departamentos de Matem\'aticas y de Computaci\'on, en la d\'ecada de los noventa. Es en esta d\'ecada en que un n\'umero creciente de instituciones reconocieron la influencia penetrante de la Computaci\'on, creando unidades independientes como departamentos, escuelas o institutos dedicados a tal \'area de estudio, un cambio que ha demostrado tanto perspicacia como previsi\'on. En Per\'u, un n\'umero cada vez mayor de instituciones de educaci\'on superior han tratado de seguir el desarrollo de las universidades extranjeras (aunque no siempre en forma muy seria o exitosa), reconociendo a la Computaci\'on como un \'area de estudio en s� misma, as\'i como su importancia estrat\'egica en la educaci\'on, y creando departamentos, escuelas o institutos dedicados a su estudio. La Facultad de \FacultadName no puede ser la excepci\'on a este cambio, en el que ya se tiene un retraso relativo con muchas de las instituciones educativas dentro y fuera de Per\'u.
