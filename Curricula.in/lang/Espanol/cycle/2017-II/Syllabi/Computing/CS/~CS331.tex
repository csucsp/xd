\begin{syllabus}

\course{CS331. Cloud Computing}{Obligatorio}{CS331}

\begin{justification}
La capacidad de procesamiento de una sola m�quina es limitada y la Ley de Moore se ha encontrado 
con barreras antes de lo previsto, a pesar de esto la necesidad de mayor poder computacional es cresciente. 

El uso de las computadoras como elementos conectados entre s� es cada vez m�s com�n y cada vez en mayor escala, 
la capacidad de comunicaci�n entre dispositivos (computadoras, celulares, pdas, etc.), abre las puertas 
a la existencia de una �nica plataforma donde la informaci�n de los usuarios
est� disponible siempre, sin importar el medio de acceso a esta (\textit{Cloud computing}).

La computaci�n en la nube de internet o un grupo de computadores 
permite conseguir ambos objetivos, traspasando la barrera de una sola m�quina para poder
integrar las capacidades de distintos dispositivos y permitirles interactuar en un entorno que
el usuario perciba como unificado; adem�s, al conectarlos, el tope de desempe�o
del sistema ya no es la capacidad de un s�lo elemento (e.g. CPU) sino la cantidad de participantes en este,
por lo cual existe una escalabilidad del poder computacional much�simo mayor.
\end{justification}

\begin{goals}
\item Conocer los elementos b�sicos del dise�o de Sistemas Distribu�dos
\item Comprender, en el contexto de sistema distribuidos, conceptos fundamentales como transparencia y tolerancia a fallos.
\item Evaluar las ventajas y 
\item Aprender a instalar y usar aplicaciones en Sistemas Distribu�dos
\end{goals}

\begin{outcomes}
\ExpandOutcome{a}{3}
\ExpandOutcome{b}{4}
\ExpandOutcome{c}{4}
\ExpandOutcome{d}{3}
\ExpandOutcome{i}{3}
\ExpandOutcome{j}{4}
\ExpandOutcome{k}{4}
\end{outcomes}

\begin{unit}{\ARDistributedArchitecturesDef}{tanenbaum1996,coulouris2005}{3}{2}
    \ARDistributedArchitecturesAllTopics
    \ARDistributedArchitecturesAllObjectives
\end{unit}

\begin{unit}{Modelos de sistema}{orfali1999,tanenbaum2003}{3}{2}
   \begin{topics}
        \item \SESpecializedSystemsTopicDistributed%
        \item \SESpecializedSystemsTopicClient%
        \item \NCIntroductionTopicClient%
        \item \SESpecializedSystemsTopicParallel%
        \item \SESpecializedSystemsTopicWeb%
   \end{topics}

   \begin{unitgoals}
        \item Entender distintos modelos de computaci�n distribuida.
   \end{unitgoals}
\end{unit}

\begin{unit}{Soporte del Sistema Operativo}{orfali1999,tanenbaum2003}{3}{3}
   \begin{topics}
      \item Procesos e Hilos.
      \item Modelos.
   \end{topics}

   \begin{unitgoals}
      \item Conocer el soporte del Sistema Operativo a los Sistemas Distribu�dos.
   \end{unitgoals}
\end{unit}

\begin{unit}{\ALDistributedAlgorithmsDef}{tanenbaum1996,coulouris2005}{3}{4}
    \ALDistributedAlgorithmsAllTopics
    \ALDistributedAlgorithmsAllObjectives
\end{unit}

\begin{unit}{\NCNetworkedApplicationsDef}{tanenbaum1996,coulouris2005}{3}{3}
    \NCNetworkedApplicationsAllTopics
    \NCNetworkedApplicationsAllObjectives
\end{unit}

\begin{unit}{Middleware}{sheldon1995,tanenbaum2003,sheldon1994}{3}{3}
   \begin{topics}
      \item Llamada a un procedimiento remoto (RPC).
      \item Middleware Orientado a Mensajes (MOM).
      \item Peer-to-Peer.
      \item Servicio de directorio.
      \item Seguridad.
   \end{topics}

   \begin{unitgoals}
      \item Entender los principios del middleware.
   \end{unitgoals}
\end{unit}

\begin{unit}{Objetos Distribu�dos: Modelos de Componentes}{tanenbaum1996,hoque1998,coulouris2005,sheldon1995}{3}{3}
   \begin{topics}
      \item Objetos y Componentes.
      \item Beneficios.
      \item Modelos de Componentes CORBA, COM, EJB, RMI.
   \end{topics}

   \begin{unitgoals}
      \item Conocer diferentes casos de objetos distribuidas.
      \item Entender los conceptos del modelo de componentes.
      \item Entender las similitudes y diferencias, ventajas y desventajas de los diferentes modelos de componentes.
   \end{unitgoals}
\end{unit}

\begin{unit}{Sistemas Distribu�dos de Archivos}{tanenbaum1996,coulouris2005}{3}{2}
   \begin{topics}
          \item \OSFileSystemsTopicFiles%
          \item \OSFileSystemsTopicFile%
          \item \OSFileSystemsTopicStandard%
          \item \OSFileSystemsTopicSpecial%
          \item \OSFileSystemsTopicNaming%
          \item Replicaci�n y Tolerancia a Fallos.
   \end{topics}

   \begin{unitgoals}
      \item Entender el funcionamiento de los Sistemas Distribu�dos de Archivos
   \end{unitgoals}
\end{unit}

\begin{coursebibliography}
\bibfile{Computing/CS/CS314}
\end{coursebibliography}

\end{syllabus}
