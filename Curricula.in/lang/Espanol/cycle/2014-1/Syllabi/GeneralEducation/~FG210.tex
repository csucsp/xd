\begin{syllabus}

\course{FG210. �tica}{Obligatorio}{FG210}

\begin{justification}
Procurarle al alumno los criterios de discernimiento general y particular, as� como las pautas morales para que con ellos oriente su conducta personal y profesional y pueda comprender y valorarlos como medio de realizaci�n integral de su persona a trav�s de actos queridos. Conscientes, libres y responsables.
\end{justification}

\begin{goals}
\item Hacer entender que no todo comportamiento es digno de la persona.
\item Facilitar las l�neas generales sobre las que debe apoyarse cualquier actuaci�n, para que sea verdaderamente humana.
\end{goals}

\begin{outcomes}
\ExpandOutcome{e}{3}
\ExpandOutcome{g}{4}
\end{outcomes}

\begin{unit}{Introducci�n a la �tica}{Gomez,Piper,Vaticano}{11}{1}
\begin{topics}
	\item La Ley Moral y su relaci�n con a persona humana.
	\item El acto humano: en que consiste; algunos principios.
	\item La conciencia: tipos de conciencia.
	\item La Libertad: diversas concepciones  de libertad, responsabilidad y derechos humanos.
\end{topics}
\begin{unitgoals}
	\item Introducir al alumno al mundo de la �tica y de los principales componentes del actuar �tico
\end{unitgoals}
\end{unit}

\begin{unit}{Segunda Unidad}{Gomez,Piper,Vaticano}{11}{1}
\begin{topics}
	\item La virtud en cuanto disposici�n humana.
	\item Las virtudes fundamentales (prudencia, justicia, fortaleza, templanza).
	\item Los vicios.
\end{topics}
\begin{unitgoals}
	\item Comprender, valorar y fortalecer la decisi�n de vivir seg�n la verdad de s� mismo y los valores objetivos, e integrar dichos valores en la toma de decisiones en su campo profesional.
\end{unitgoals}
\end{unit}

\begin{unit}{Tercera Unidad}{Gomez,Piper,Vaticano}{11}{1}
\begin{topics}
	\item �tica y sexualidad.
	\item Matrimonio
	\item Vida Humana
	\item Problem�tica de la vida humana
\end{topics}
\begin{unitgoals}
	\item Abordar los principios problemas que tocan a la �tica especial y proporcionar criterios de discernimiento.
\end{unitgoals}
\end{unit}

\begin{unit}{Cuarta Unidad}{Gomez,Piper,Vaticano}{12}{1}
\begin{topics}
	\item Sentido �tico de la profesi�n
	\item Virtudes relacionadas al ejercicio profesional
	\item Deontolog�a;: revisi�n de c�digos profesionales.
	\item Problem�tica �tica tratada por Comit�s de �tica de Colegios Profesionales. Casos m�s comunes.
\end{topics}
\begin{unitgoals}
	\item Comprender, valorar y fortalecer la decisi�n de vivir seg�n la verdad de si mismo y los valores objetivos, e integrar dichos valores en ala toma de decisiones en su campo profesional.
\end{unitgoals}
\end{unit}

\begin{coursebibliography}
\bibfile{GeneralEducation/FG101}
\end{coursebibliography}
\end{syllabus}
