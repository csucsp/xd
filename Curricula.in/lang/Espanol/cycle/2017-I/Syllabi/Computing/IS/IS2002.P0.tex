\begin{syllabus}

%\curso{\IS2002.P0Def}{\IS2002.P0Type}
\curso{\IS2002.P0. Productividad Personal con Tecnolog�a de Sistemas de Informaci�n}{Obligatorio}{CS240S}

\begin{justification}
Los estudiantes aprender�n a utilizar tecnolog�as de informaci�n con el objetivo de mejorar
la produci�n tanto de individuos o grupos dentro de las organizaciones.
\end{justification}

\begin{goals}
\item Entender y aplicar conceptos  sobre la productividad del personal de las organizaciones haciendo uso de herramientas inform�ticas.
\end{goals}

\begin{outcomes}
\ExpandOutcome{a}
\end{outcomes}

\begin{unit}{Hojas de c�lculo}{Excel2007Walkenbach}{8}
   \begin{topics}
      \item Conceptos sobre la productividad en el trabajo.
      \item Utilizaci�n b�sica de una hoja de c�lculo.
      \item Creaci�n y manipulaci�n de plantillas
      \item Creaci�n y manipulaci�n de macros.
   \end{topics}

   \begin{unitgoals}
      \item Conocer y aplicar distinas  tegnolog�as  de software con el fin de mejorar la productividad en el personal de las organizaciones.
   \end{unitgoals}
\end{unit}

\begin{unit}{Manipulaci�n de datos}{}{8}
   \begin{topics}
      \item Mantenimiento y organizaci�n de los datos (Ordenaci�n y filtros) mediante el uso de herramientas como gestores de base de datos y hoja de datos.
      \item Acceso a datos organizacionales y externos.
   \end{topics}

   \begin{unitgoals}
      \item  Desarrollar conocimientos sobre dise�o y implementaci�n de Base de Datos as� como t�cnicas de acceso a datos para mejorar la eficiencia de respuesta de los procesos en los trabajadores.
   \end{unitgoals}
\end{unit}

\begin{unit}{Acceso a datos}{}{12}
   \begin{topics}
      \item Estrategias de b�squeda de informaci�n.
      \item Herramientas de optimizaci�n y personalizaci�n.
   \end{topics}

   \begin{unitgoals}
      \item Entender y conocer distintas estrat�gias de b�squeda con el fin de dotar a los alumnos la capacidad para disernir entre una estrategia  u otra dependiendo el tipo de problema a tratar.
      \item  Conocer herramientas de sortware para optimizar y personalizar trabajos tanto individuales como en equipo.
   \end{unitgoals}
\end{unit}

\begin{unit}{Dise�o de p�ginas Web}{}{24}
   \begin{topics}
      \item Dise�o de documentos profesionales.
      \item Dise�o de P�ginas Web.
      \item Construcci�n  y Dise�o de Presentaciones  efectivas 
    \end{topics}
  \begin{unitgoals}
      \item Desarrollar en los alumnos la capacidad para dise�ar  e implimentar documentos y presentaciones con 
       con calidad profesional.
  \end{unitgoals}
\end{unit}

\begin{unit}{Dise�o de presentaciones}{}{24}
   \begin{topics}
      \item Construcci�n  y Dise�o de Presentaciones  efectivas 
    \end{topics}
  \begin{unitgoals}
      \item Desarrollar en los alumnos la capacidad para dise�ar  e implimentar documentos y presentaciones con 
       con calidad profesional.
  \end{unitgoals}
\end{unit}

\bibfile{Computing/IS/IS-Bib}

\begin{coursebibliography}














































































\end{coursebibliography}

\end{syllabus}
