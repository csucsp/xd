\begin{syllabus}

\course{CS367. Rob�tica}{Electivo}{CS367}

\begin{justification}
Que el alumno conozca y comprenda los conceptos y principios fundamentales de control, planificaci�n de caminos 
y definici�n de estrat�gias en rob�tica m�bil as� como conceptos de percepci�n rob�tica de forma que entienda 
el potencial de los sistemas rob�ticos actuales.
\end{justification}

\begin{goals}
\item Sistentizar el potencial y las limitaciones del estado del arte de los sistemas toboticos actuales.
\item Implementar algoritmos de planeamiento de movimientos simples
\item Explicar las incertezas asociadas con sensores y la forma de tratarlas
\item Dise�ar una arquitectura de control simple
\item Describir v�rias estrat�gias de navegaci�n
\item Entender el rol y las aplicaciones de la percepci�n rob�tica
\item Describir la importancia del reconocimiento de imagenes y objetos en sistemas inteligentes
\item Delinear las principales t�cnicas de reconocimiento de objetos
\item Describir las diferentes caracter�sticas de las tecnolog�as usadas en percepci�n
\end{goals}

\begin{outcomes}
\ExpandOutcome{a}{3}
\ExpandOutcome{i}{4}
\ExpandOutcome{TASDSH}{4}
\end{outcomes}

\begin{unit}{\ISRoboticsDef}{Thrun2005,Siegwart2004}{30}{4}
   \ISRoboticsAllTopics
   \ISRoboticsAllObjectives
\end{unit}

\begin{unit}{\ISPerceptionDef}{Gonzales2007,Sonka2007}{30}{3}
   \ISPerceptionAllTopics
   \ISPerceptionAllObjectives
\end{unit}



\begin{coursebibliography}
\bibfile{Computing/CS/CS367}
\end{coursebibliography}

\end{syllabus}
