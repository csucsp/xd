\begin{syllabus}

\course{FG107. Fundamentos Antropol�gicos de la Ciencia}{Obligatorio}{FG107}

\begin{justification}
La formaci�n profesional tiene efectos importantes en la conducta de la persona humana, 
toda vez que el desarrollo de sus capacidades, le permite una manera de desplegarse y 
de relacionarse consigo mismo, con los dem�s y con el mundo creado.  
La posibilidad de caer en reduccionismos antropol�gicos y de tener una visi�n 
inadecuada de al persona humana en su desempe�o profesional, hace necesario 
ahondar y reflexionar en una correcta aproximaci�n al ser humano as� como 
realizar un an�lisis profundo del compromiso personal, social y moral que 
implica el ser profesional.
\end{justification}

\begin{goals}
\item Que el estudiante comprenda que el estudio del hombre es fundamental para la actividad profesional.
\item Que la actividad profesional no se vea afectada por reduccionismos de diversos tipos.
\end{goals}

\begin{outcomes}
\ExpandOutcome{FH}{2}
\ExpandOutcome{HU}{3}
\ExpandOutcome{TASDSH}{3}
\end{outcomes}

\begin{unit}{Visi�n del Hombre como Fundamento de la Actividad Profesional}{Lepp63,Lepp64,Figari98,Figari04,Doig}{0}{2}
\begin{topics}
	\item ?`Qui�n soy? 
	\item ?`Por qu� existo?.  
	\item Necesidad del hombre de comprenderse.  
	\item El misterio del hombre y su conocimiento.  
	\item Un ser contingente con hambre de infinito.  
	\item Ser bio-psico-espiritual.
	\item Dinamismos fundamentales.  
	\item Naturaleza y dignidad del ser humano.  
	\item El hombre como sujeto:  ser �nico e irrepetible. 
	\item Autoconciencia autodeterminaci�n. 
	\item El sentido de la vida y el ser profesional. 
	\item El hombre: un ser para el encuentro.  
	\item Desarrollo de la perspectiva din�mica del ser humano como persona y como profesional: 
	      relaci�n con uno mismo, relaci�n con los dem�s, relaci�n con el mundo creado, relaci�n con el fundamento de mi vida.  
	\item El problema del mal: el profesional y las rupturas, las 3 concupiscencias, la reconciliaci�n.
	\item Ser persona y ser profesional.  
	\item Conceptos b�sicos sobre la Libertad.  
	\item Libertad de elecci�n y libertad de acto.  
	\item La Recta acci�n.  
	\item La Prudencia aplicada a la toma de decisiones.  
	\item La relevancia de las decisiones empresariales y la moral: la persona jur�dica y la toma de decisiones.  
	\item Actos moralmente buenos y malos.
\end{topics}

\begin{unitgoals}
	\item Entender al hombre como un todo como fundamenteo para el ejercicio profesional.
\end{unitgoals}
\end{unit}

\begin{unit}{Reduccionismos Antropol�gicos}{Lepp63,Lepp64,Figari98,Figari04,Doig}{0}{2}
\begin{topics}
	\item Los reduccionismos antropol�gicos: determinismo, psicologismo, economicismo, marxismo, liberalismo, comunismo, estatismo, cientificismo.  
	\item Exposici�n y critica de los principales antropolog�as de nuestro tiempo. 
	\item El fin ultimo del hombre y el fin ultimo de la organizaci�n o empresa.
	\item El trabajo como �mbito de realizaci�n del hombre.
\end{topics}

\begin{unitgoals}
	\item Entender al hombre como un todo y no de forma parcial o incompleta.
\end{unitgoals}
\end{unit}

\begin{unit}{Persona y Cultura}{Lepp63,Lepp64,Figari98,Figari04,Doig}{0}{2}
\begin{topics}
	\item El hombre como creador de cultura: la cultura personal y la cultural organizacional.
	\item El profesional como agente de cambio cultural. 
	\item Cultura y anti-cultura en el ambito laboral.
	\item ?`Qu� es la Tecnolog�a? Tecnolog�a y mundo actual. 
	\item Impacto de la tecnolog�a en la cultura actual. 
	\item La mentalidad tecnologista. Dimensi�n antropol�gica y cultural de la tecnolog�a. 
	\item ?`Es neutral la tecnolog�a? Persona, cultura y productos tecnol�gicos. 
	\item Consideraciones �ticas relativas a la tecnolog�a.
\end{topics}

\begin{unitgoals}
	\item Entender al hombre como creador e cultura.
\end{unitgoals}
\end{unit}

\begin{unit}{El Compromiso Social de la Iglesia}{Lepp63,Lepp64,Figari98,Figari04,Doig}{0}{1}
\begin{topics}
	\item El Desarrollo integral y los valores humanos: aproximaci�n humana y social a los valores. 
	\item La solidaridad. 
	\item La subsidiaridad. 
	\item El desarrollo integral: el hombre como centro del desarrollo. 
	\item La pobreza y la responsabilidad social de la empresa: qu� es ser pobre. 
	\item Pobreza y dignidad humana. 
	\item El compromiso social de ser profesional. 
	\item La responsabilidad social de la empresa, el compromiso social del due�o de los factores de producci�n.
\end{topics}

\begin{unitgoals}
	\item Entender y poder poner en pr�ctica el compromiso social del la Iglesia para el beneficio del hombre y de la sociedad.
\end{unitgoals}
\end{unit}

\begin{coursebibliography}
\bibfile{GeneralEducation/FG101}
\end{coursebibliography}

\end{syllabus}
