\begin{syllabus}

\curso{CM251. �lgebra Lineal}{Obligatorio}{CM251}

\begin{justification}
En este curso se estudiar�n los espacios vectoriales, determinantes, transformaciones lineales, �lgebra multilineal.
Autovalores y formas can�nicas, operadores sobre espacios con producto interno formas bilineales y cuadr�ticas. Geometr�a Af�n y Transformaciones afines.
\end{justification}

\begin{goals}
\item  Lograr que el alumno asimile los conceptos b�sicos sobre espacios vectoriales, transformaciones lineales, matrices, as� como determinantes y sus aplicaciones
\item  Dotar al estudiante de los conocimientos b�sicos de temas de �lgebra Lineal que son de utilidad para el estudio de otros cursos y sus aplicaciones
\end{goals}

\begin{outcomes}
\ExpandOutcome{a}
\ExpandOutcome{i}
\ExpandOutcome{j}
\end{outcomes}

\begin{unit}{Espacios Vectoriales en general}{Halmos58}{6}
   \begin{topics}
         \item  Definici�n y ejemplos
	 \item  Subespacios, sus propiedades. Suma y suma directa
         \item  Independencia lineal, base y dimensi�n
	 \item  Producto interno. Bases ortogonales; ortogonalizaci�n de Gram-Schmidt
         \item  Distancia de un punto a una variedad lineal. (Aplicaci�n a la Geometr�a)
         \item  El espacio cociente
   \end{topics}

   \begin{unitgoals}
         \item  Entender los conceptos y caracter�sticas de los espacios vectoriales
         \item  Resolver problemas
   \end{unitgoals}
\end{unit}

\begin{unit}{Transformaciones lineales}{Halmos58,Hoffman71}{8}
   \begin{topics}
         \item  Definici�n y ejemplos.
	 \item  Teorema fundamental de las transformaciones lineales y sus consecuencias.
         \item  �lgebra de las transformaciones lineales. Espacio de las transformaciones lineales. Espacio dual.
	 \item  Matrices. Sus operaciones. Rango e inversa. Matriz asociada a una transformaci�n lineal. Matrices equivalentes y semejantes.
         \item  Autovalores y autovectores. Forma triangular. Teorema de Cayley-Hamilton. Forma racional y de  Jordan. Transformaciones lineales diagonalizables, criterios.
	 \item  Tipos especiales de matrices: Sim�tricas, antisim�tricas, unitaria y ortogonal. Su diagonalizaci�n.
   \end{topics}

   \begin{unitgoals}
         \item  Entender los conceptos y caracter�sticas de las Transformaciones lineales
         \item  Resolver problemas
   \end{unitgoals}
\end{unit}

\begin{unit}{Determinantes}{Lages95}{6}
   \begin{topics}
         \item  Funci�n determinante.
	 \item  Propiedades.
         \item  Existencia y Unicidad del determinante
	 \item  C�lculo del determinante y determinante de una transformaci�n lineal.
         \item  Cofactores, menores y adjuntos.
	\item Determinante y rango de una matriz. Aplicaciones.
   \end{topics}

   \begin{unitgoals}
         \item  Entender los conceptos y caracter�sticas de los determinantes
         \item  Resolver problemas
   \end{unitgoals}
\end{unit}

\begin{unit}{�lgebra Multilineal}{Lang90}{8}
   \begin{topics}
         \item  Aplicaciones bilineales.
	 \item  Productos tensoriales.
         \item  Isomorfismos can�nicos.
	 \item  Producto tensoriales de aplicaciones lineales.
         \item  Cambio de coordenadas de un tensor.
	 \item  Producto tensorial de espacios vectoriales.
         \item  �lgebra tensorial de un espacio vectorial.
   \end{topics}

   \begin{unitgoals}
         \item  Entender y aplicar los conceptos del �lgebra Multilineal
         \item  Resolver problemas
   \end{unitgoals}
\end{unit}

\begin{unit}{Autovalores y formas can�nicas}{Chavez05}{8}
   \begin{topics}
	\item  Valores y vectores propios.
	\item  Triangulaci�n de matrices. El Teorema Cayley-Hamilton
	\item  Criterios de diagonalizaci�n.
	\item  Matrices nilpotentes.
	\item Forma can�nica de Jordan.
	\item La exponencial de una matriz.
   \end{topics}

   \begin{unitgoals}
         \item  Entender y aplicar los conceptos de Autovalores y formas can�nicas.
         \item  Resolver problemas.
   \end{unitgoals}
\end{unit}

\begin{unit}{Operadores sobre espacios con producto interno}{Nomizu66}{6}
   \begin{topics}
	\item  La adjunta de un operador.
	\item  Matrices positivas.
	\item  Isometr�as.
	\item  Proyecci�n perpendicular.
	\item  Operadores autoadjuntos. El Teorema Espectral.
	\item  Operadores normales.
	\item Funciones definidas sobre transformaciones lineales.
   \end{topics}

   \begin{unitgoals}
         \item  Entender y aplicar los conceptos de Operadores sobre espacios con producto interno
         \item  Resolver problemas
   \end{unitgoals}
\end{unit}

\begin{unit}{Formas bilineales y cuadr�ticas}{Kaplansky74}{8}
   \begin{topics}
	\item Formas bilineales.
	\item Suma directa y diagonalizaci�n.
	\item El teorema de inercia.
	\item Teorema de cancelaci�n de Witt.
	\item Planos hiperb�licos, formas alternadas.
	\item Witt equivalencia.
	\item Formas hermitianas.
   \end{topics}

   \begin{unitgoals}
         \item Entender y aplicar los conceptos de Formas bilineales y cuadr�ticas.
         \item Resolver problemas.
   \end{unitgoals}
\end{unit}

\begin{unit}{Geometr�a afin}{Chavez05}{6}
   \begin{topics}
         \item  Planos afines.
	 \item  Planos proyectivos.
         \item  Transformaciones proyectivas.
	 \item  Raz�n doble.
         \item  C�nicas.
         \item  Espacios de dimensi�n superior.
   \end{topics}

   \begin{unitgoals}
         \item  Entender y aplicar los conceptos de Geometr�a afin
         \item  Resolver problemas
   \end{unitgoals}
\end{unit}

\begin{coursebibliography}
\bibfile{BasicSciences/CM251}
\end{coursebibliography}

\end{syllabus}
