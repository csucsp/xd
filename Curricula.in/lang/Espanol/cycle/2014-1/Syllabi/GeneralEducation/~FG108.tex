\begin{syllabus}

\course{FG108. Antropolog�a Cultural}{Obligatorio}{FG108}

\begin{justification}    
El presente curso busca introducir al estudiante en la metodolog�a y los temas fundamentales del 
objeto de estudio de la Antropolog�a Cultural. Aprender a conocer al hombre a trav�s de sus 
creaciones culturales, comprender su dimensi�n cultural, su actitud de apertura y 
tolerancia hacia otra culturas.
\end{justification}

\begin{goals}
\item Formar en el estudiante la capacidad de observaci�n y an�lisis del comportamiento humano a lo largo de la historia  
\item Comprender las l�gicas de funcionamiento de los elementos culturales propios de las sociedades y cosmovisiones de los pueblos
\end{goals}

\begin{outcomes}
\ExpandOutcome{f}{1}
\ExpandOutcome{FH}{1}
\end{outcomes}

\begin{unit}{Antropolog�a y Cultura}{Herskovitz84}{12}{1}
   \begin{topics}
      \item La Antropolog�a como disciplina
	\item Teor�a y M�todo de la Antropolog�a Cultural
	\item El concepto de cultura. La Cultura Ideal y la Cultura Real
	\item El contexto cultural del comportamiento humano. Exclusi�n e Inclusi�n. Justicia y Violencia.
	\item Manifestaciones de la Violencia
   \end{topics}

   \begin{unitgoals}
      \item Definir el objeto de estudio y comprender la importancia de la Antropolog�a Cultural como una disciplina indispensable para poder obtener conocimiento y capacidad de an�lisis del comportamiento humano
   \end{unitgoals}
\end{unit}

\begin{unit}{Cultura, Sociedad e Individuo}{Nanda82}{10}{1}
   \begin{topics}
      \item El hombre y la Cultura
	\item Aprendizaje de la Cultura. Transculturaci�n, aculturaci�n y endoculturaci�n
	\item Rango y estratificaci�n social
	\item Matrimonio, Familia y grupos dom�sticos. Concepci�n Machista. Feminismo en el siglo XX
	\item Parentesco asociaci�n. Las Relaciones con la Familia
   \end{topics}

   \begin{unitgoals}
      \item Explicar la importancia de la cultura en la comprensi�n del hombre como ser social
   \end{unitgoals}
\end{unit}

\begin{unit}{Expresi�n Simb�lica}{Marvin97}{10}{1}
   \begin{topics}
      \item Cultura y Visi�n del mundo. Cultura de Paz
	\item Expresiones Culturales: Lenguaje y Arte
	\item Religi�n, Fen�meno Religioso, Mito, Ritual
	\item La Medicina Tradicional y la enfermedad
	\item Cultura, Religi�n y Salud
   \end{topics}

   \begin{unitgoals}
      \item Comprender y analizar los distintos tipos de expresiones culturales como aspecto importante e inseparable del ser humano
   \end{unitgoals}
\end{unit}

\begin{coursebibliography}
\bibfile{GeneralEducation/FG108}
\end{coursebibliography}
\end{syllabus}
