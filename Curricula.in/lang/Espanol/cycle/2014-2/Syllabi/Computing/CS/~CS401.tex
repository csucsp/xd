\begin{syllabus}

\course{CS401. Proyecto I}{Obligatorio}{CS401}

\begin{justification}
Este curso tiene por objetivo que el alumno aprenda a realizar una
investigaci�n de car�cter cient�fico en el �rea de computaci�n.
\end{justification}

\begin{goals}
\item Que el alumno aprenda como se inicia una investigaci�n cient�fica en el �rea de computaci�n.
\item Que el alumno conozca las principales fuentes para obtener bibliograf�a relevante para trabajos de investigaci�n en el �rea de computacion: Researchindex, IEEE-CS\footnote{http://www.computer.org}, ACM\footnote{http://www.acm.org}.
\item Que el alumno pueda redactar documentos t�cnicos en computaci�n utilizando \LaTeX.
\item Que el alumno sea capaz de \underline{reproducir} los resultados ya existentes en un determinado t�pico a trav�s de la experimentaci�n.
\item Los entregables de este curso son:
	\begin{description}
	\item [Avance parcial:] Dominio del tema del art�culo y bibliograf�a preliminar en formato de art�culo \LaTeX.
	\item [Final:] Entendimiento del art�culo, documento conclu�do donde se contenga los resultados experimentales de la(s) t�cnica(s) estudiada(s).
	\end{description}
\end{goals}

\begin{outcomes}
\ExpandOutcome{a}{3}
\ExpandOutcome{b}{3}
\ExpandOutcome{c}{3}
\ExpandOutcome{e}{4}
\ExpandOutcome{f}{3}
\ExpandOutcome{h}{3}
\ExpandOutcome{i}{4}
\ExpandOutcome{l}{3}
\end{outcomes}

\begin{unit}{Iniciaci�n cient�fica en el �rea de computaci�n}{ieee,acm,citeseer}{60}{5}
  \begin{topics}
      \item B�squeda bibliogr�fica en computaci�n.
      \item Redacci�n de art�culos t�cnicos en computaci�n.
  \end{topics}
  \begin{unitgoals}
      \item Aprender a hacer una investigaci�n correcta en el �rea de computaci�n
      \item Conocer las fuentes de bibliograf�a adecuada para esta �rea
      \item Saber redactar un documento de acorde con las caracter�sticas que las conferencias de esta �rea exigen
  \end{unitgoals}
\end{unit}

\begin{coursebibliography}
\bibfile{Computing/CS/CS401}

\end{coursebibliography}
\end{syllabus}
