\newcommand{\DocumentVersion}{V2.0}
\newcommand{\fecha}{10 de Marzo de 2010}
\newcommand{\YYYY}{2010\xspace}
\newcommand{\Semester}{2010-1\xspace}
\newcommand{\city}{Arequipa\xspace}
\newcommand{\country}{Per�\xspace}
\newcommand{\dictionary}{Espa�ol\xspace}
\newcommand{\GraphVersion}{2\xspace}
\newcommand{\CurriculaVersion}{2\xspace}
\newcommand{\OutcomesList}{a,b,c,d,e,f,g,h,i,j,k,l,m,HU,FH,TASDSH}
\newcommand{\logowidth}{7cm}

\newcommand{\University}{Universidad Nacional de San Agust�n\xspace}
\newcommand{\InstitutionURL}{http://www.unsa.edu.pe\xspace}
\newcommand{\underlogotext}{}

\newcommand{\FacultadName}{Ingenier�a de Producci�n y Servicios\xspace}
\newcommand{\DepartmentShortName}{Ingenier�a de Sistemas e Inform�tica\xspace}
\newcommand{\SchoolFullName}{Escuela Profesional de Ingenier�a de Sistemas\xspace}
\newcommand{\SchoolFullNameBreak}{Facultad de \FacultadName \\ \SchoolFullName}
\newcommand{\SchoolShortName}{Ingenier�a de Sistemas\xspace}
\newcommand{\SchoolAcro}{CC\xspace}

\newcommand{\GradoAcademico}{Bachiller en Ingenier�a de Sistemas\xspace}
\newcommand{\TituloProfesional}{Ingeniero de Sistemas\xspace}
\newcommand{\GradosyTitulos}%
{\begin{description}%
\item [Grado Acad�mico: ] \GradoAcademico y%
\item [Titulo Profesional: ] \TituloProfesional%
\end{description}%
}
\newcommand{\SchoolURL}{http://www.unsa.edu.pe\xspace}
\newcommand{\doctitle}{Plan curricular \YYYY de la \SchoolFullName{\Large\footnote{\SchoolURL}}\xspace}
\newcommand{\AbstractIntro}{Este documento es el proyecto de creaci�n de la \SchoolFullName de la \University.\xspace}

\newcommand{\OtherKeyStones}{}

\newcommand{\profile}{%
El perfil profesional de este programa profesional puede ser mejor entendido a partir de
\OnlyMainDoc{la Fig. \ref{fig.cs} (P�g. \pageref{fig.cs})}\OnlyPoster{las figuras del lado derecho}.
Este profesional tiene una fuerte orientaci�n al �mbito empresarial y utiliza a la computaci�n 
como un medio para lograr su objetivo.

El profesional en Sistemas de Informaci�n tiene por objetivo hacer que las organizaciones locales y globales existentes funcionen a la velocidad 
del siglo XXI gracias a la correcta aplicaci�n de tecnolog�a computacional. 
Nuestra formaci�n profesional tiene 3 pilares fundamentales: 
Formaci�n Humana, 
un contenido de acuerdo a normas internacionales y 
una orientaci�n marcada a la organizaci�n.
}

\newcommand{\HTMLFootnote}{Generado por <A HREF='http://socios.spc.org.pe/ecuadros/'>Ernesto Cuadros-Vargas</A> <ecuadros AT spc.org.pe><BR>basado en el modelo de la <A HREF='http://www.spc.org.pe/'>Sociedad Peruana de Computaci�n</A> y en la Computing Curricula de <A HREF='http://www.acm.org/'>ACM</A>/<A HREF='http://www.aisnet.org/'>AIS</A>}

\newcommand{\Copyrights}{Generado por Ernesto Cuadros-Vargas (ecuadros AT spc.org.pe), Sociedad Peruana de Computaci�n (http://www.spc.org.pe/), Universidad Cat�lica San Pablo (http://www.ucsp.edu.pe) \country basado en la {\it Computing Curricula} de IEEE-CS (http://www.computer.org) y ACM (http://www.acm.org/)}
